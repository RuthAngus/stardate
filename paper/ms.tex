\documentclass[useAMS, usenatbib, preprint, 12pt]{aastex}
% \documentclass[a4paper,fleqn,usenatbib,useAMS]{mnras}
\usepackage{cite, natbib}
\usepackage{float}
\usepackage{epsfig}
\usepackage{cases}
\usepackage[section]{placeins}
\usepackage{graphicx, subfigure}
\usepackage{color}
\usepackage{bm}

\newcommand{\columbia}{3}
\newcommand{\cca}{2}
\newcommand{\amnh}{1}
\newcommand{\kepler}{{\it Kepler}}
\newcommand{\corot}{{\it CoRoT}}
\newcommand{\Ktwo}{{\it K2}}
\newcommand{\ktwo}{\Ktwo}
\newcommand{\TESS}{{\it TESS}}
\newcommand{\tess}{{\it TESS}}
\newcommand{\LSST}{{\it LSST}}
\newcommand{\lsst}{{\it LSST}}
\newcommand{\Wfirst}{{\it WFIRST}}
\newcommand{\wfirst}{{\it WFIRST}}
\newcommand{\SDSS}{{\it SDSS}}
\newcommand{\PLATO}{{\it PLATO}}
\newcommand{\plato}{{\it PLATO}}
\newcommand{\Gaia}{{\it Gaia}}
\newcommand{\gaia}{{\it Gaia}}
\newcommand{\panstarrs}{{\it PanSTARRS}}
\newcommand{\Teff}{$T_{\mathrm{eff}}$}
\newcommand{\teff}{$T_{\mathrm{eff}}$}
\newcommand{\FeH}{[Fe/H]}
\newcommand{\feh}{[Fe/H]}
\newcommand{\fhat}{$\hat{F}$}
\newcommand{\pmega}{$\bar{\omega}$}
\newcommand{\mj}{$m_j$}
\newcommand{\mh}{$m_h$}
\newcommand{\mk}{$m_k$}
\newcommand{\mx}{$m_x$}
\newcommand{\ie}{{\it i.e.}}
\newcommand{\eg}{{\it e.g.}}
\newcommand{\logg}{log \emph{g}}
\newcommand{\dnu}{$\Delta \nu$}
\newcommand{\numax}{$\nu_{\mathrm{max}}$}
\newcommand{\gcolor}{$G_{Bp} - G_{Rp}$}

\newcommand{\racomment}[1]{{\color{red}#1}}

\begin{document}

% \title{Inferring stellar ages by combining isochrone fitting with
% gyrochronology}
\title{A new age-dating model for cool main sequence stars that combines
stellar evolution models with gyrochronology}
% \title{A new age-dating model for cool main sequence stars that combines
% stellar evolution models with gyrochronology}

% \author{%
%     Ruth Angus\altaffilmark{\amnh, }\altaffilmark{\cca, }\altaffilmark{\columbia},
%    {\it et al.}
% }

% \altaffiltext{\amnh}{American Museum of Natural History, Central Park West,
% Manhattan, NY}
% \altaffiltext{\cca}{Center for Computational Astrophysics, Flatiron Institute,
% 162 5th Avenue, Manhattan, NY}
% \altaffiltext{\columbia}{Department of Astronomy, Columbia
% University, NY, NY}

% This paper should not be a calibration exercise -- stick to your guns and
% keep it simple!
% Switch to a pure isochrone fitting approach for hot stars and evolved stars
% for now.

\begin{abstract}
    We present a new age-dating technique that combines gyrochronology and
isochrone fitting to infer ages for FGKM main-sequence and subgiant field
stars.
The age-dating methods of gyrochronology and isochrone fitting are each
capable of providing relatively precise ages in certain areas of the
Hertzsprung-Russell diagram: rotation periods can provide precise ages for
cool stars on the main sequence via gyrochronology, and isochrone fitting
can provide precise ages for stars near the main-sequence turnoff.
Combined, these two age-dating techniques can be applied to a broader range of
stellar masses and evolutionary stages and can provide ages that are more
precise and accurate than either method used in isolation.
In this investigation, we demonstrate that the position of a star on the
Hertzsprung-Russell or color-magnitude diagram can be combined with its
rotation period to infer a precise age via both isochrone fitting and
gyrochronology simultaneously.
We show that incorporating rotation periods with 5\% uncertainties into
stellar evolution models particularly improves age precision for FGK stars on
the main sequence, and can, on average, provide age estimates that are up to
three times more precise than isochrone fitting alone for these stars.
In addition, we provide a new gyrochronology relation, calibrated to the
Praesepe cluster and the Sun, that includes a variance model to capture the
rotational behavior of stars whose rotation periods do not lengthen with the
square-root of time, and parts of the Hertzsprung-Russell diagram where
gyrochronology has not been calibrated.
This publication is accompanied by an open source {\it Python} package (\sd:
\url{ https://github.com/ruthangus/stardate }) for inferring the ages of main
sequence and subgiant FGKM stars from rotation periods, spectroscopic
parameters and/or apparent magnitudes and parallaxes.

\end{abstract}

% \label{section:intro}
% Motivation and context
%-----------------------------------------------------------------------------
%   - The need for better stellar ages.
%   - Why are MS ages harder than red giant ages?
%   - Introduce gyrochronology
%   - How does gyrochronology work?
%   - Theoretical vs empirical gyro models
%   - Motivation for improvements to gyrochronology relations
%   - Describe the project presented here.
%   - Sum up
%   - Paper outline
% Stalled. I really need to pin down the unique difference between this method
% and Jen van Saders model.

% Motivation and context
% Relevant literature
% Paper outline

% Motivation and context
%-----------------------
The formation and evolution of the Milky Way (MW) and the planetary systems
within it are two topics of significant interest to the astronomical community
today.
Both of these fields require precise and accurate ages of thousands of stars.
% In order to study the formation of the MW, it is necessary to infer the ages
% of its constituent parts, its stars.
Advances in galactic archaeology have recently been made using the ages of
red giant stars, some derived from asteroseismology and some from
spectroscopy, to probe the age distribution of the MW.
Red giants are highly luminous and can be observed to great distances, thus
providing age information on the scale of tens of kilo-parsecs.
Main sequence stars, although fainter are more numerous and their ages may
provide new insights into the formation and evolution of the Solar
neighborhood.
Stellar ages are also of great interest for studying the formation and
evolution of planetary systems.
Almost all exoplanets discovered to date orbit main sequence (MS) stars and it
is therefore the ages of MS stars that are needed to capture snapshots of
planet evolution.
Unfortunately, the very property that makes MS stars good hosts for habitable
planets also makes them difficult to date: they do not change substantially
over time.

Stellar ages provide the key to understanding the evolution of all
astrophysical objects.
For main sequence (MS) stars however, age is a difficult property to infer.
This is predominantly because hydrogen burning stars do not change appreciably
during their time spent on the MS: a star like the Sun will grow in
luminosity by around a factor of two before turning off the MS.
In addition, the Sun's temperature will only increase by around 100 K during
its $\sim$8 billion year MS lifetime.
Luminosity and temperature are therefore not sensitive proxies for age.
On the other hand, Sun's rotation period will vary by almost an order of
magnitude over its MS lifetime.
Stellar rotation periods are much more sensitive to age than luminosity or
temperature.
Ages inferred using isochrone fitting use the fact that stars get brighter and
hotter over time.
Incorporating rotation period measurements into isochrone fitting methods
provides additional information that allows for much more precise age
inference.
The models developed and calibrated by \citet{epstein2014, vansaders2015,
vansaders2016} are stellar evolution models which use rotation period as an
additional parameter and the methodology presented here is related to these
models but uses an empirically calibrated gyrochronology model, as opposed to
a physically derived one.

In addition to the difficulties imposed by the slow timescale for variability
within MS stars, different dating methods often produce inconsistent
predictions for the age of a star.
For example, an asteroseismic age will not necessarily agree with a isochronal
or rotational age.
This is in part because the underlying processes generating the evolution of
the observable properties are different and in part because our understanding
of the underlying physics is flawed or incomplete.
The various available dating methods can be categorised by the underlying
physical process they trace.
For example, evolutionary models track the radial extent of the
hydrogen-burning core and age-rotation relations model the evolving state of
the internal magnetic dynamo.
In addition, dating methods can be classified by their level of empiricism,
\ie\ the number of free parameters that need to tuned when fitting the models
to the data.
The physics behind the evolving luminosity and effective temperature of a star
as a result of core hydrogen burning is, for example, very well understood and
does not need calibrating; physics determine these models.
On the other hand, magnetic activity evolution is poorly understood and must
be calibrated using available data.
In table \ref{tab:dating_methods} we provide an overview of various dating
methods, the main observables associated with them, the underlying physics
driving the changing observables, the types of star the method applies to and
the empirical or physical nature of the model.

\subsection{Rotation-Dating}
\label{sec:rotation}

Main sequence (MS) stars comprise the majority of our galaxy but their ages
are notoriously difficult to measure.
Their positions on the HR diagram don't change significantly during their
    hydrogen burning lifetimes, a fact that is convenient for life on Earth
    but inconvenient for galactic archaeologists.
Now, due to the abundance of rotation periods for MS stars provided by Kepler
    and to-be provided by TESS, LSST and Wfirst, rotation-dating is the most
    readily available, precise method for inferring stellar ages.
Rotation-dating works well for young stars but a question mark still hangs
    over its accuracy for stars older than the Sun.
Recent results show that old \kepler\ asteroseismic stars rotate more rapidly
    than expected given their age \citep[\eg][]{Angus2015, Vansaders2016,
    Metcalfe2016}.
This has been attributed to an evolving magnetic dynamo: as stars reach a
    critical Rossby number (the ratio of rotation period to the convective
    overturn timescale), their magnetic field `switches off' and stars
    maintain a consistent rotation period after that time.
Whilst this physical explanation produces a model that fits the data, it
    is driven by observations, not theory, and other explanations could
    provide an answer.
The data sets typically used to test the age-rotation relations are highly
    heterogeneous and each set has its own detection and selection biases.
For example, asteroseismology favours quiet stars whereas rotation periods are
    easiest to measure for active stars.

% A history of rotation-dating.
The phenomenon of magnetic braking in MS stars was first observed almost fifty
years ago by \citet{Skumanich1972} who observed that the rotation periods of
the Sun and young cluster stars seemed to decay with the square-root of time.
Later, a mass-dependence was added to the relation between age and rotation
period --- less massive stars lose angular momentum faster than more massive
ones.
\citet{Kawaler1988} derived a formalism for this angular momentum loss and his
relation depended on the mass loss rate, the ....
More recently, \citet{Barnes2003} demonstrated that a simple relation could be
used to describe `gyrochronology', the method of rotation-dating, and further
works \citep[\eg][]{Barnes2007, Mamajek2008, Barnes2010, Meibom2011},
continue to demonstrate that the relation between rotation period and age
holds true while theorists \citep[\eg][]{Matt2012, Epstein2014} modify and
extend the efforts to produce physical models of this phenomenon.

Not only do a number of dating methods exists, several different models are
often available for the same dating method.
In the case of rotation-dating...

\subsection{Stellar Evolution models}
The ongoing physical processes in the core of a star is reflected externally
by an increase in luminosity and temperature.
As hydrogen is converted to helium via nuclear fusion in the core, the mass
fraction, etc, etc.
Leading to etc, etc.



% \section{Method}
% \label{section:method}
% The model
%-----------------------------------------------------------------------------
%   - Introduction
%   - Why iso and gyro are independent
%   - A description of the PGM
%   - The PGM
%   - The formulas
%   - A description and reiteration in words.
%   - Priors
%   - The gyrochronology models
%   - The Praesepe model
% Practicalities: sampling, etc.
%-----------------------------------------------------------------------------
%   - isochrones.py - readthedocs
%   - Step-by-step description
%   - Emcee, including assessing convergence.
\section{Method}
\label{section:method}

%   - Introduction
In this section we describe our combined isochrone fitting and gyrochronology
model.
A common approach to stellar age-dating is to make separate age predictions
using separate sets of observables.
For example, if a star's rotation period, parallax, and apparent magnitudes in
a range of bandpasses are available, it is possible to predict its age from
both gyrochronology and isochrone fitting separately.
How these two age predictions are later combined is then a difficult choice.
Is it best to average these predictions, to use the more precise of the two,
or the one believed to be more accurate?
The methodology described here provides an objective method for combining age
estimates.
There is, after all, only one age for each star.
Combining information from different models can be relatively simple, as long
as the processes being modeled; those that generated the data, are
independent.
In this case, we are combining information that relates to the burning of
hydrogen in the core (this is the process that drives the slow increase in
\teff\ and luminosity over time) with information about the magnetic braking
history of a star (the current rotation period).
We can assume that, to first order, these two processes are independent: the
hydrogen fraction in the core does not affect a star's rotation period and
vice versa.
In practise,  we can never be entirely sure that two such processes are
independent but, at least within the uncertainties, any dependence here is
unlikely to affect our results.
If this assumption is valid, the likelihoods calculated using each model can
be multiplied together.

The desired end product of this method is an estimate of the non-normalized
posterior probability density function (PDF) over the age of a star,
\begin{equation} \label{eqn:eqn1}
    p(A|{\bf m_x}, T_{\mathrm{eff}}, \log(g), \hat{F},
    P_{\mathrm{rot}}, \bar{\omega}),
\end{equation}
where $A$ is age, ${\bf m_x}$ is a vector of
apparent magnitudes in various bandpasses, \fhat\ is the {\it observed} bulk
metallicity, \prot\ is the rotation period and \pmega\ is parallax.
In order to calculate a posterior PDF over age, we must marginalize over
parameters that relate to age, but are not of interest in this study: mass
($M$), distance ($D$), V-band extinction ($A_V$) and the {\it inferred} bulk
metallicity, $F$.
The marginalization involves integrating over these extra parameters,
\begin{eqnarray} \label{eqn:bayes}
    & p(A|{\bf m_x}, T_{\mathrm{eff}}, \log(g), \hat{F},
    P_{\mathrm{rot}}, \bar{\omega})
\\ \nonumber
    & \propto \int p({\bf m_x}, T_{\mathrm{eff}}, \log(g), \hat{F},
    P_\mathrm{rot}, \bar{\omega}|
    A, M, D, A_V, F)~p(A)p(M)p(D)p(A_V)p(F)dMdDdA_VdF.
\end{eqnarray}
This equation is a form of Bayes' rule,
\begin{equation} \label{eqn:eqn2}
\mathrm{Posterior} \propto \mathrm{Likelihood} \times \mathrm{Prior},
\end{equation}
where the likelihood of the data given the model is,
\begin{equation} \label{eqn:full_likelihood}
    p({\bf m_x}, T_{\mathrm{eff}}, \log(g), \hat{F}, \bar{\omega},
    P_{\mathrm{rot}}|A, M, D, A_V, F),
\end{equation}
and the prior PDF over parameters is,
\begin{equation} \label{eqn:prior}
    p(A)p(M)p(D)p(A_V)p(F).
\end{equation}

%   - Why iso and gyro are independent.
Not all of the observables on the left of the `$|$' in the likelihood depend
on all of the parameters to the right of it.
For example, rotation period, \prot\ does not depend on V-band extinction,
$A_V$.
In our model, we make use of conditional independencies like this and use them
to factorize the likelihood.
Instead of the likelihood of equation \ref{eqn:full_likelihood},
where every observable depends on every parameter, our model can be factorized
as,
\begin{equation} \label{eqn:factorized}
    p({\bf m_x}, T_{\mathrm{eff}}, \log(g), \hat{F}, \bar{\omega},
    C_{B-V}|A, M, D, A_V, F) ~p(P_\mathrm{rot}|A, C_{B-V}),
\end{equation}
where we have introduced a new parameter, $C_{B-V}$, which is the $B-V$ color
that is often used as a mass proxy in the literature.
In our model $C_{B-V}$ is not measured but {\it inferred}: it is a latent
parameter.
We infer $C_{B-V}$ because many stars do not have a directly measured $B-V$
color.
For example, most \kepler\ stars have {\it 2MASS} photometry in J, H and K
bands and \Gaia\ photometry in $G$, $G_{BP}$ and $G_{RP}$, but do not all have
B and V band colors.
However, the gyrochronology model we use is calibrated to B-V color, not J-K
or otherwise \citep{barnes2007, mamajek2008, angus2015}.
A probabilistic graphical model (PGM) depicting the joint probability over
parameters and observables is shown in figure \ref{fig:PGM}.
It describes the conditional dependencies between parameters (in white
circles) and observables (in grey circles) with arrows leading from the causal
processes to the dependent processes.
For example, it is the mass, age, metallicity, extinction and distance that
determines the observed spectroscopic properties (\teff, \logg, \feh)
and apparant magnitudes, ${\bf m_x}$).
These parameters also determine the \cbv\ color of a star.
In turn, it is a star's age and \cbv\ color that determine its rotation
period.
Note that, written this way, stellar rotation periods do not directly depend
on stellar mass.
Mass, age and metallicity determine $C_{B-V}$, and $C_{B-V}$ along with age
determines rotation period.
The purpose of this PGM is not to depict the physical realities of stellar
evolution, it is only a visual description of the structure of the model we
use here.
Breaking up the problem this way allows us to efficiently join isochronology
and gyrochronology and infer the joint age of a star from all its observables.
It may well be that rotation period depends directly on mass and metallicity
in reality, but it is more practical for us to assume that these dependencies
are weak enough not to significantly affect the ages that we ultimately infer.

%   - The formulas
The factorization of the likelihood described in equation \ref{eqn:factorized}
and depicted in figure \ref{fig:PGM} allows us to multiply two separate
likelihood functions together: one computed using an isochronal model and one
computed using a gyrochronal model.
We assume that the probability of observing the measured observables, given
the model parameters is a Gaussian and that the observables are identically
and independently distributed.
These assumptions allow us to use Gaussian likelihood functions.
The isochronal likelihood function is,
\begin{eqnarray} \label{eqn:isochrones_only_likelihood}
    & \mathcal{L_{\mathrm{iso}}} = p({\bf m_x}, T_{\mathrm{eff}}, \log(g),
    \hat{F},
    \bar{\omega}, C_{B-V}|A, M, D,
    A_V, F) \\ \nonumber
    & = \frac{1}{\sqrt{(2\pi)^n \det(\Sigma)}}
    \exp\left( -\frac{1}{2} ({\bf O_I} - {\bf I})^T \Sigma ^{-1}
    ({\bf O_I} - {\bf I})\right),
\end{eqnarray}
where ${\bf O_I}$ is the n-dimensional vector of $n$ observables: \teff,
\logg, \fhat, \pmega, ${\bf m_x}$ (where $n$ is $4 + $ the number of
apparant magnitudes in different pass-bands that are available) and $\Sigma$
is the covariance matrix of that set of observables.
${\bf I}$ is the vector of {\it model} observables that correspond to a set of
parameters: $A$, $M$, $F$, $D$ and $A_V$, calculated using an isochrone model.
We assume there is no covariance between these observables and so this
covariance matrix consists of individual parameter variances along the
diagonal with zeros everywhere else.
The gyrochronal likelihood function is,
\begin{eqnarray} \label{eqn:gyro_likelihood}
    & \mathcal{L_{\mathrm{gyro}}} = p(P_\mathrm{rot} |A, C_{B-V}) \\ \nonumber
    & = \frac{1}{\sqrt{(2\pi) \det(\Sigma_P)}}
    \exp\left( -\frac{1}{2} ({\bf P_O} - {\bf P_P})^T \Sigma ^{-1}
    ({\bf P_O} - {\bf P_P})\right),
    % = \prod_i \frac{1}{\sqrt{2\pi}\sigma_i} \exp
    % \left(-\frac{(P_{\mathrm{obs}, i} - P_{\mathrm{pred},
    % i})^2}{2\sigma_i^2}\right),
\end{eqnarray}
% where $P_{\mathrm{obs}, i}$ is the $i$th observed rotation period,
% $P_{\mathrm{pred}, i}$ is the corresponding predicted rotation period,
% calculated from the $i$th age and $C_{B-V}$ values predicted by the isochronal
% model.
where ${\bf P_O}$ is a 1-D vector of observed rotation periods, ${\bf P_P}$ is
the vector of corresponding predicted rotation periods, calculated using the
vector of ages and $C_{B-V}$ values that correspond to the input parameters
as predicted by the isochronal model.
The full likelihood function used in our model is the product of these two
likelihood functions,
\begin{eqnarray} \label{eqn:full_likelihood}
    & \mathcal{L_{\mathrm{full}}} = \frac{1}{\sqrt{(2\pi)^n \det(\Sigma)}}
    \exp\left( -\frac{1}{2} [{\bf O_I} - {\bf I}]^T \Sigma ^{-1}
    [{\bf O_I} - {\bf I}]\right) \\ \nonumber
    & \times
    \frac{1}{\sqrt{(2\pi) \det(\Sigma_P)}}
    \exp\left( -\frac{1}{2} [{\bf P_O} - {\bf P_P}]^T \Sigma ^{-1}
    [{\bf P_O} - {\bf P_P}]\right).
\end{eqnarray}

%   - Priors
We place priors over the model parameters $A$, $M$, $F$, $D$ and $A_V$.
These priors represent our `prior beliefs' about the values these parameters
will take, before we use the data to update those beliefs via a likelihood and
produce a `posterior' belief about their values.
These priors are described in the appendix.

% Practicalities: sampling, etc.
%-----------------------------------------------------------------------------
%- isochrones.py
To calculate ${\bf I}$, the vector of predicted isochronal observables, we use
the {\tt isochrones.py} {\it python} package which has a range of
functionalities relating to isochrone fitting.
The first of the {\tt isochrones.py} functions we use is the likelihood
function of equation \ref{eqn:isochrones_only_likelihood}.
The {\tt isochrones.py} likelihood function accepts a dictionary of
observables which can, but does not {\it have} to include, all of the
following: \teff, \logg, $F$, parallax and apparent magnitudes in a range of
colors, as well as the uncertainties on all these observables.
It then calculates the residual vector $({\bf O_I} - {\bf I})$ where ${\bf
O_I}$ is the vector of observables and ${\bf I}$ is a vector of corresponding
predicted observables.
The prediction is calculated using a set of isochrones \citep[we use the MIST
models,][]{paxton2011, paxton2013, paxton2015, dotter2016, choi2016, paxton2018},
where the set of {\it model} observables that correspond
to a set of physical parameters is returned.
This requires interpolation over the model grids since, especially at high
dimensions, it is unlikely that any set of physical parameters will exactly
match a precomputed set of isochrones.
The observables that correspond to a set of physical parameters go into ${\bf
I}$ and the {\tt isochrones.py} likelihood function returns the result of
equation \ref{eqn:isochrones_only_likelihood}.
The second {\tt isochrones.py} function we use is one that predicts \cbv\ for
a given set of stellar parameters.
This color is then used to calculate the gyrochronal likelihood function of
equation \ref{eqn:gyro_likelihood}.

%   - Step-by-step description
The inference processes procedes as follows (as a reminder, we use {\it
observables} to refer to the data: \teff, \logg, etc and {\it parameters} to
refer to the model parameters: age, mass, distance, etc).
First, a set of parameters: age, mass, true bulk metallicity, distance and
extinction, as well as observables \teff, \logg, bulk metallicity, apparent
magnitudes and parallax (${\bf O_I}$) for a single star are passed to the
isochronal likelihood function, equation
\eqref{eqn:isochrones_only_likelihood}.
Then, a set of {\it model} values of \teff, \logg, bulk metallicity, apparent
magnitudes and parallax (${\bf I}$) that correspond to that set of parameters
are calculated by {\tt isochrones.py}.
The isochronal log-likelihood, $\ln(\mathcal{L}_{\mathrm{iso}})$, is then
computed for these parameter values.
The same age that was passed to the likelihood function, and the $C_{B-V}$
corresponding to it, along with the observed rotation period, are then passed
to the gyrochronal likelhood function (equation \ref{eqn:gyro_likelihood}).
The gyrochronal log-likelihood, $\ln(\mathcal{L}_{\mathrm{gyro}})$, is
computed.
The full log-likelihood is then calculated,
\begin{equation} \label{eqn:both_likelihood}
\ln(\mathcal{L}_{\mathrm{full}})
= \ln(\mathcal{L}_{\mathrm{iso}}) + \ln(\mathcal{L}_{\mathrm{gyro}}),
\end{equation}
and added to the log-prior to produce a single sample from the posterior PDF.

%   - Emcee, including assessing convergence.
When applying our model to infer the age of a star, we sampled the joint
posterior PDF over age, mass, metallicity, distance and extinction using the
affine invariant ensemble sampler, {\tt emcee} \citep{foreman-mackey2013} with
24 walkers.
Samples were drawn from the posterior PDF until 100 {\it independent} samples
are obtained.
We actively estimated the autocorrelation length, which indicates how many
steps are taken per independent sample, after every 100 steps using the
autocorrelation tool built into {\tt emcee}.
The MCMC concluded when {\it either} 100 times the autocorrelation length was
reached and the change in autocorrelation length over 100 samples was less
than 0.01, {\it or} the maximum of 100,000 samples was obtained.
This method is trivially parallelizable, since the inference process for each
star can be performed on a separate core.
The age of a single star can be inferred in around one hour on a laptop
computer.

%   - The gyrochronology models
The gyrochronology model we used to predict $P_P$ is, % $P_{\mathrm{pred}, i}$ is,
\begin{equation}
    P_\mathrm{rot} = A^\eta \alpha (C_{B-V} - \delta)^\beta,
\label{eqn:gyro}
\end{equation}
where \prot\ is rotation period in days, \cbv\ is a star's $B-V$ color, $A$ is
stellar age in Myrs and $\eta$, $\alpha$, $\beta$ and $\delta$ take values
0.55, 0.4, 0.31 and 0.45 respectively \citep{angus2015}.
This functional form was introduced by \citep{barnes2007} and the parameter
values are adopted from the recalibration performed in \citet{angus2015},
which is based on young cluster stars and old asteroseismic stars.
Because this gyochronology model is not calibrated to include stars turning
off the main sequence whose rotation slows rapidly due to their increasing
radius, we do not apply gyrochronology to stars with a MIST model \eep\
greater than 425\footnote{This number was determined by investigating the
correspondance between \eep\ and position on the HR diagram.
See the {\it Jupyter} notebook at
\url{https://github.com/RuthAngus/stardate/blob/master/paper/code/EEP_cutoff.ipynb}.
In future, \eep\ could be used as a parameter in our gyrochronology model.
}.
In addition, stars with \cbv $>$ 0.45, corresponding to a temperature around
6250 K are only modeled using isochrones, not gyrochronology.
These stars have thin convective envelopes and do not spin down substantially
over their \ms\ lifetimes so their rotation periods do not strongly predict
their ages.
However, isochrone fitting can provide relatively precise ages for these hot
stars, as well as evolved stars with large \eep s.

It was recently shown that a simple power law in age does not provide a good
fit to old asteroseismic stars \citep{angus2015, vansaders2016}.
It is hypothesized that the magnetic braking of these old stars has ceased and
cannot be modeled with a Skumanich-like spin-down law \citep{vansaders2016}.
In future, the above model could and should be updated to include a more
flexible treatment of rotation period as a function of age in order to account
for the change of slope in the relation.
Until then, this method should only be used for stars with Rossby number below
2.1 \citep{vansaders2016}, \ie\ their ratio of rotation period to convection
overturn time ($P/\tau = Ro$) does not exceed 2.1.
In this work we are chiefly concerned with introducing a new framework where
rotation periods are modeled {\it simultaneously} with isochronal features.
Although the gyrochronology models used here do not provide a good fit to all
the available data, we reiterate that no single model {\it is} able to
reproduce all the data, and that there is utility in using such a simple,
linear, empirical model like this.
Again, we are not attempting to improve gyrochronology models in this work: in
this paper we are more concerned with introducing a new approach to modeling
stellar ages, however, our method is highly flexible and modular and an
improved gyrochronology model could easily be swapped in for this one in
future.
Our model would allow a linear combination of other, {\it physical} parameters
to be used to predict age from rotation period, like $\log g$, metallicity and
mass.
In future, it may be better to model stars in physical rather than observable
parameter space.

% Although the gyrochronology model described above \citep[equation
% \ref{eqn:gyro},][]{angus2015} has been calibrated using a number of cluster
% stars, it does not provide a good fit to any individual cluster.
% No current gyrochronology model is able to capture the behavior of rotation as
% a function of color and age for individual benchmark clusters: the shape of
% this relation is different in each and current models are not flexible enough
% to capture inter-cluster differences in rotational evolution.
% For this reason, we also explored the rotational evolution of a single
% cluster, in order to produce a best-case model and demonstrate the potential
% of rotation-dating in a case where the model is perfectly accurate.
% We chose Praesepe as it is a relatively old open cluster \citep[$\sim$ 600
% Myrs][]{gossage2018}, meaning its Solar-type members have converged onto the
% rotational main sequence, and it is relatively compact on the sky so many of
% its members were observed during a single \ktwo\ campaign.
% In fact, Praesepe was repeatedly observed by \ktwo, in Campaigns 5, 16 and 18,
% however we only use rotation periods published from the analysis of Campaign 5
% in this work \citep{rebull2016}.

% % The Praesepe model
% We used a three-dimensional polynomial model to predict rotation period as a
% function of \gaia\ color and age for Praesepe and the Sun.
% This model consists of a 4th order polynomial in logarithmic Gaia color:
% $G_{Bp} - G_{Rp}$, which we write as $C_G$ for simplicity, and a 1st order
% polynomial (a straight line) in logarithmic age.
% We used \gcolor\ instead of (B-V) because, due to the $\sim$ billion stars
% observed by \gaia, it is now the most abundant and widely available
% photometric color.
% Our gyrochronology likelihood function is designed to compare observed
% rotation period to predicted rotation period.
% For this reason the gyrochronology model we used must predict rotation period
% as a function of age and color.
% However, when {\it calibrating} the gyrochronology model, we chose to make
% {\it age} the dependent variable because the uncertainties on age are much
% greater than the uncertainties on rotation period.
% Since we are using a linear model, the relation is easily invertable.
% We fitted the following model to Praesepe members:
% \begin{equation}
%     \log_{10}(A) = a + b\log_{10}(C_G) + c\log_{10}^2(C_G) +
%     d\log_{10}^3(C_G) + e\log_{10}^4(C_G) + f\log_{10}(P)
% \label{eqn:gyro_age_praesepe}
% \end{equation}
% where $P$ is rotation period in days, $C_G$ is Gaia color, $A$ is stellar age
% in years and the lower case letters are free parameters which we fitted to the
% data using linear least squares.
% We adopted an age for Praesepe of 600 million years \citep{gossage2018}, a
% Solar age of 4.56 Gyr \citep{connelly2012}, and a Solar rotation period of 26
% days \citep[][Morris \etal, in prep]{balthasar1986, howe2000}.
% The Sun's color in the Gaia color bandpasses, $G_{Bp} - G_{Rp}$, is 0.82
% \citep{casagrande2018}.
% We found best-fit values: $a = 7.37 \pm 0.03, b = -1.4 \pm 0.1, c = 5.0 \pm
% 0.8, d = -34 \pm 3, e = 66 \pm 14$, and $f = 1.49 \pm 0.02$.
% Rotation periods for Prasepe were obtained from \citet{rebull2017} and their
% \gaia\ colors were obtained by crossmatching their sky-projected positions
% with the \gaia\ DR2 catalog.
% % The $f$ parameter is the inverse of the slope of the rotation period and age
% % which was originally measured to be around 0.5.
% % Our Praesepe and Sun-only fit results in a slightly steeper age dependence of
% % around 0.67, however this value is likely be
% We inverted this relation to predict rotation period as a function of color
% and age,
% \begin{equation}
%     \log_{10}(P) = \frac{\log_{10}(A) - a - b\log_{10}(C_G) - c\log_{10}^2(C_G) -
%     d\log_{10}^3(C_G) - e\log_{10}^4(C_G)}{f}.
% \label{eqn:gyro_age_praesepe}
% \end{equation}
% Both gyrochronology models of equations \ref{eqn:gyro} and
% \ref{eqn:gyro_age_praesepe} are used to predict the ages of individual
% Praesepe stars from their rotation periods and apparent magnitudes in section
% \ref{section:results}.

% The PGM
\begin{figure}
  \caption{
A probabilistic graphical model (PGM) showing the conditional
dependencies between the parameters (white nodes) and
observables (gray nodes) in our model.
% ${\bf \theta}$ is a vector of {\it parameters}: mass, observed bulk
%     metallicity, distance and V-band extinction; and ${\bf O}$ is a vector of
%     {\it observables}: apparent magnitudes, $m_x$, effective temperature,
%     \teff, surface gravity, \logg, observed bulk metallicity, $\hat{F}$, and
%     parallax, $\bar{\omega}$.
% determined by the mass, $M$, age, $A$, distance, $D$, extinction, $A_V$
% and bulk metallicity, $F$, of a star.
${\bf \theta}$ is a vector of {\it parameters}: mass, observed bulk
    metallicity, distance and V-band extinction; and ${\bf O}$ is a vector of
    {\it observables}: apparent magnitudes, effective temperature, surface
    gravity, observed bulk metallicity, and parallax.
    The observables, ${\bf O}$, are determined by the parameters, $A$ and
    ${\bf \theta}$.
    $C_{B-V}$ is a latent parameter that is also determined by the parameters
    $A$ and ${\bf \theta}$.
    In our model, the rotation period observable, $P$, is determined {\it
    only} by the age, $A$, and color ${C_{B-V}}$ parameters.
The dependencies of observables on parameters and parameters on parameters are
    indicated by arrows that start at a `parent' node and end at the dependent
    observable, or `child' node.
In our model, rotation period does not directly depend on distance,
extinction, metallicity or mass, only age and B-V color.
This PGM is a representation of the factorized joint PDF over parameters and
observables of equation \ref{eqn:factorized}.
}
  \centering
    \includegraphics[width=.7\textwidth]{PGM}
\label{fig:PGM}
\end{figure}


% \section{Results}
% \label{section:results}
% Intro
% TEST 1: simulations
%-----------------------------------------------------------------------------
%   - The simulated data
%   - The simulation figures
%   - The results
%   - We're cheating here.
%   - Summarize and discuss the experiment.
% TEST 2: Clusters
%-----------------------------------------------------------------------------
%   - The cluster data
%   - Praesepe results: ages
%   - Praesepe results: metallicity, mass, distance
%   - Praesepe results: the Praesepe gyro relation

%   - Summary

% TEST 3: Asteroseismology
%-----------------------------------------------------------------------------
%   - The asteroseismic data
%   - The results

% TEST 4: with and without spectroscopy on simulated data.
\section{Results}
\label{section:results}

% Intro
% TEST 1: simulations
%-----------------------------------------------------------------------------
%   - The simulated data
%   - The results
% PLOT: histograms over inferred ages (1) just isochrones (2) just
% gyrochronology, (3) both.

% TEST 2: Clusters
%-----------------------------------------------------------------------------
%   - The cluster data
%   - The results
% PLOT: histograms over inferred ages (1) just isochrones (2) just
% gyrochronology, (3) both.

% Rolling out/future
%   - Adding new datasets/dimensions/methods.
%-----------------------------------------------------------------------------

% Intro
%-----------------------------------------------------------------------------
In order to demonstrate the functionality of our method, we conducted two
tests.
In the first we simulated a set of observables from a set of fundamental
parameters for a few hundred stars using the MIST \citep{choi} stellar
evolution models.
In the second we tested our model by attempting the measure the ages of stars
in the Praesepe open cluster who's age has been measured precisely because it
is an ensemble of coeval stars with the same metallicity; a single stellar
poplulation, and its age can be established through isochrone fitting and MS
turn-off.

% TEST 1: simulations
%-----------------------------------------------------------------------------
%   - The simulated data
For the first test we began with a set of 1000 stars and drew masses, ages,
bulk metallicities, distances and extinctions at random from the following
uniform distributions:
\begin{eqnarray}
& M \sim U(0.5, 1.5)~[M_\odot] \\
& A \sim U(0.5, 14)\mathrm{~[Gyr]} \\
& F \sim U(-0.2, 0.2) \\
& D \sim U(10, 1000)~\mathrm{[pc]} \\
& A_V \sim U(0, 1).
\end{eqnarray}
\teff, \logg, \fhat, {\bf \mx}, \pmega\ and B-V were then generated using
these stellar parameters with the MIST stellar evolution models \citep{choi}
and rotation periods, $P$ were generated from the gyrochronology relation in
equation \ref{eqn:gyro} with age, $A$, and B-V.
% Uncertainties.
We then performed cuts on these simulated stars to remove evolved stars and
stars that are too hot.
The rotation periods of evolved stars, defined here to be those with \logg\ >
4.5 begin to increase as soon as they turn off the MS and their radii start to
enlarge and cannot be modeled with the gyrochronology relation of equation
\ref{eqn:gyro_age}.
In addition, hot stars (defined as 6250 K < \teff) cannot be modeled using
equation \ref{eqn:gyro_age} because their convective envelopes are extremely
shallow and their magnetic fields are weaker, leading to a lack of magnetic
braking.
The rotation periods of these stars do not increase substantially during their
time on the MS.
After performing these cuts, 649 \racomment{update} stars remained in the
sample of simulated stars.
% We attempted to measure the ages of the stars in this sample using the method
% outlined in section \ref{sec:method}.
We took two approaches to inferred the ages of these simulated stars: firstly
using {\it only} a stellar evolution model, and secondly using a stellar
evolution model {\it combined with} a gyrochronology model.
For all stars, our initial guesses for the parameters are $M = 1M_\odot$, $A =
1$ Gyr, $F = 0$, $D = 500$ pc and $A_V = 0.1$.

% The simulation figures

% iso only
\begin{figure}
  \caption{
The results of a test in which we simulated observable properties of stars
    with the same model we used to infer their properties.
In this experiment we used {\it only} stellar evolution models to infer ages;
we did not use rotation periods.
    For results where we used {\it both} stellar evolution models {\it and}
    gyrochronology, see figure \ref{fig:iso_and_gyro}.
The true age, used to produce associated observables is shown on the x-axis,
    and the ages we inferred are shown on the y-axos.
This figure shows the posterior PDFs over stellar age for each of the
    simulated stars as a `violin plot', where samples from the posterior are
    plotted vertically as a smooth, symmetric function.
The widths of these functions indicates the probability over age: wider
    regions represent more probable ages.
The median values of the posterior PDFs are plotted as solid horizontal lines.
This figure demonstrates that when only stellar evolution models are used to
    infer ages for field MS stars, the resulting predicted ages are extremely
    imprecise.
}
  \centering
    \includegraphics[width=1\textwidth]{../plots/iso_only_violin.pdf}
\label{fig:iso_only}
\end{figure}

\begin{figure}
  \caption{
The results of a test in which we simulated observable properties of stars
    with the same model we used to infer their properties.
    In this experiment we used {\it both} stellar evolution models to {\and}
    rotation periods to infer ages.
For results where we used stellar evolution models {\it only}, see the
    previous figure (figure \ref{fig:iso_only}).
The true age, used to produce associated observables is shown on the x-axis,
    and the ages we inferred are shown on the y-axis.
This figure shows the posterior PDFs over stellar age for each of the
    simulated stars as a `violin plot', where samples from the posterior are
    plotted vertically as a smooth, symmetric function.
The widths of these functions indicates the probability over age: wider
    regions represent more probable ages.
The median values of the posterior PDFs are plotted as solid horizontal lines.
    This figure demonstrates that when rotation periods (gyrochronoloy) {\it
    and} stellar evolution models are used to infer the ages of field MS
    stars, the resulting predicted ages relatively precise; much more precise
    than when using stellar evolution models alone.
}
  \centering
    \includegraphics[width=1\textwidth]{../plots/iso_and_gyro_violin.pdf}
\label{fig:iso_and_gyro}
\end{figure}

%   - The results
Figure \ref{fig:iso_only} shows the results of using a
stellar evolution model model to estimate the posterior PDFs over the stellar
ages of simulated stars.
The rotation periods of stars have not been incorporated into this model:
these posterior PDFs were obtained by isochrone fitting only, using the
likelihood function in equation \ref{eqn:iso_likelihood}.
In most cases ages are only weakly constrained by the stellar evolution
models.
In some cases there is no constraint on the stellar age: the age of the star
is consistent with all ages from 0-14 Gyrs.
The reason for this is that the temperatures and luminosities of stars do not
change very much on the main sequence.
The isochrones are tightly spaced in the MS region of the HR-diagram and, as a
result, even precisely measured temperatures and luminosities often do not
yield precise ages.
Figure \ref{fig:iso_and_gyro} shows the results of using a stellar evolution,
combined with a gyrochronology model.
These ages have been inferred using the likelihood of equation
\ref{eqn:both_likelihood}.
Again, the true stellar ages are plotted on the $x$-axis and the posterior
PDFs of the inferred ages are plotted on the $y$-axis.
Here, unlike the case where only stellar evolution models were used, the
recovered ages are precise.
This is because gyrochronology isochrones (or gyrochrones) are more widely
separated relative to the observational uncertainties than the isochrones used
above.
Put another way, the rotation periods of two stars of different ages and the
same mass will have rotation periods that differ significantly -- almost
certainly more than the observational uncertainty on rotation period.
On the other hand, two stars of the same mass and different age are likely to
have extremely similar luminosities and temperatures and the differences
between these properties are likely to be smaller than the observational
uncertainties.

% We're cheating here.
Figure \ref{gyro_only} demonstrates the results of inferring ages using
rotation periods only, and this illustrates why the combination of isochronal
and gyrochronal ages is so precise: almost all this precision comes from
rotation periods.
This simulation experiment is unrealistic for two main reasons: firstly, we
simulated data from the same gyrochronology model we used to infer ages and so
the results will be perfectly accurate by design.
Secondly, we simulated data without any intrinsic scatter built into the
gyrochronology model; it is a deterministic model.
This means that a rotation period and color predicts a corresponding
single-valued age, rather than a probability distribution over ages.
This is unrealistic given observations of open clusters who's members clearly
show excess scatter in their rotation periods, particularly for less massive
stars.
These results look precise and accurate, but this is misleading.
Inaccuracies would arise if the gyrochronology model was incorrect or poorly
calibrated in all parts of parameter space and imprecision would arise if
intrinsic scatter were built into the gyrochronology model.
The result of using a deterministic model, such as the one used in this
experiment, is that the uncertainties on stellar ages will be unrealistically
small.

%   - Summarize and discuss the experiment.
In this experiment, we compared the precision of MS field star ages inferred
with stellar evolution models only, and with stellar evolution models {\it
combined} with gyrochronology models.
We showed that including gyrochronology in the stellar evolution model results
in much more precise age predictions.
We have not yet made any statement about accuracy however; the above
experiment produces accurate ages by construction.
In order to test the potential of this method to produce accurate results, we
test our model on real data in the following section.

% TEST 4: with and without spectroscopy
% Although inferred ages are likely to be more precise if spectroscopic
% parameters are available (\teff, \logg\ and \feh), it is still possible to
% place constraints on stellar ages if only photometric colors are available,
% especially if the star has a precise parallax measurement.
% Figure \ref{fig:just_photometry} demonstrates the decrease in precision when
% only photometric colors (J, H and K), parallaxes and rotation periods are
% used as observables.
% The top panel of figure \ref{fig:just_photometry} shows an isochrones-only
% model and the bottom panel shows the results of using isochrones {\it and}
% gyrochronology.
% When spectroscopic parameters are not available, including age information
% from a stellar rotation period becomes extremely important as it is far more
% age-sensitive than photometric colors.
% The trade-off, however, is that stellar ages will become gyrochronology
% dominated, and it is even more important to use an accurate gyrochronology
% model.

% TEST 2: Clusters
%-----------------------------------------------------------------------------
%   - The cluster data
In order to test our model on real stars with known ages, we selected a sample
of cluster stars with precisely measured ages from ensemble isochrone fitting
and main sequence turn off.
The ages of open clusters can be measured much more precisely than field
stars for two main reasons.
Firstly, the stars have the same age (to within a few million years), so the
age of a cluster can be inferred with an increased precision that is
proportional to the square root of the number stars, relative to a single star
case.
In addition, stars in the same cluster form (we assume) from the same
molecular cloud and therefore have the same metallicity.
Since cluster stars have the same metallicity and age, stars fall on the same
isochrone and the main sequence turn
off can be identified.
We compiled rotation periods and Gaia photometry and parallaxes for members of
Praesepe, a 650 Myr cluster.
We chose Praesepe because it is relatively tightly clustered on the sky and
many of its members were therefore targeted in a single \ktwo\ campaign, from
which it was possible to measure rotation periods via frequency analysis of
member's light curves \citep{douglas2015}.
% We compiled rotation periods and spectroscopic parameters for members of
% the Pleiades which is 150 million years old, Praesepe, a 650 Myr cluster,
% NGC 6811, 1.1 Myrs and NGC 6819, 2.5 Myrs.
% Rotation periods from the Pleiades \citep{rebull2016} and Praesepe
% \citep{douglas2016} were obtained from frequency analysis of \ktwo\ data, and
% rotation periods for NGC 6811 \citep{meibom2013} and 6819 \citep{meibom2015}
% stars were measured from \kepler\ prime light curves.
We crossmatched N Praesepe members with measured rotation periods
\citep{douglas2015}, with the Gaia DR2 catalog \racomment{Gaia DR2 citation},
using a 5'' search radius.
The result was a sample of N stars with rotation periods, parallaxes and
\gaia\ $G$, $G_{BP}$ and $G_{RP}$ -band photometry.
Figure \ref{figure:praesepe} shows the rotation periods of Praesepe members as
a function of their dust-corrected \gaia\ \gcolor\ colors.
We used the {\tt dustmaps} {\tt python} package to calculate dust extinction
along the line of sight to these stars.
The filled blue circles show the FGK stars on the `rotational main sequence'
that were used to calibrate a relation between rotation period and color for
this cluster.
% REDDENNING!!!!!

% Cutting outliers and limiting color range.
In order to fit a period-color relation to these data we restricted the sample
of cluster stars to the color range, 0.56 $<$ \gcolor\ $<$ 3 in order to
remove early F and late M dwarfs whos' rotation periods do not fall on the
`gyrochronology main sequence'.
Although it {\it may} be possible to crudely predict the ages of these stars
(at least the M dwarfs) from their rotation periods, the age-rotation-color
relation for these stars is very different to the FGK star relations and is
not the focus of this paper.
In addition, we removed rapidly rotating stars from the sample since, although
modeling outliers is important and consequential for gyrochronology in
general, the goal of this paper is not to produce a perfect gyrochronology
that reproduces stochasticity in the data, just a simple function that fits
the rotational main sequence of Praesepe.
In future we plan to update the gyrochronology models to include M dwarfs,
and the outlying rapid rotators using a mixture of Gaussians.
Similarly, we used only Praesepe in this study because the period-color
relations of each open cluster with rotation periods has a different shape.
This is likely due to differences in metalicities, incomplete or noisy
membership lists and differences in rotation period measurement algorithms.
A global gyrochronology calibration, using all cluster data is planned for the
future but, again, is beyond the scope of the project presented here.
The rotation periods of the Praesepe members in the restricted color range and
with outliers removed are plotted against their \Gaia\ colors in figure
\ref{fig:Praesepe}.
We used linear least squares to fit a linear model to Praesepe and the Sun.
A 4th order polynomial in $\log(G_{BP} - G_{RP})$ and a straight line in
$\log(\mathrm{Age~[yrs]})$ was fit to reproduce
$\log(\mathrm{rotation~period~[days]})$.
\begin{equation}
    \log(P) = a + b\log(C_G) + c\log^2(C_G) + d\log^3(C_G) + e\log^4(C_G) +
    f\log(A),
\end{equation}
\label{eq:new_gyro}
were $P$ is period, $C_G$ is \Gaia \gcolor\ color, and A is age.

% Out of a total of N Hyades stars with rotation periods \citep{radick1987,
% radick1995, hartman2011, delorme2011, douglas2016}, N of them have precise
% spectroscopic parameters \citet{brewer}.
% The rest have Gaia photometry ($G$, $G_BP$ and $G_RP$ bands), Kepler
% photometry ($Kp$) and Gaia parallaxes.
% We only selected stars with Gaia DR2 radial velocity (RV) measurements, and
% required that their RVs had to be consistent with the cluster RV of around 40
% kms$^{-1}$.
% Two stars with RVs $< 30$ kms$^{-1}$ were removed from the catalog.

% The Cluster figure
\begin{figure}
  \caption{
% The rotation periods of stars in open clusters, and the Sun, plotted against
%     their \Gaia\ colors (\gcolor) in logarithmic space.
% The result of this fit is plotted on top of the data at the ages of
%     the cluster stars and the Sun.
    The rotation periods of Praesepe members and the Sun, plotted against
    their \Gaia\ colors (\gcolor) in logarithmic space.
    We fit a 4th order polynomial to these data in order to predict
    rotation periods from \gaia\ colors, and a 1st order polynomial (a
    straight line) in age.
The result of this fit is plotted on top of the data at the ages of
    Praesepe and the Sun.
}
  \centering
    \includegraphics[width=1\textwidth]{clusters.pdf}
\label{fig:clusters}
\end{figure}

The age of each Praesepe member was inferred using both gyrochronology models:
the legacy model of equation \ref{eq:gyro} and the Praesepe model of equation
\ref{eq:new_gyro}.
This exercise is not designed to test one gyrochronology model against
another: the model fit to the Praesepe data will provide a better age
prediction by design, as the legacy model was fit to number of clusters at
once.
The point of this exercise is to show how precise gyrochronology could be if
the perfect model is used.
We did not force the cluster members to have the same age since the aim of
this experiment was to reveal the precision and accuracy of our method by
quantifying the level of scatter in our predicted ages and identifying regions
of parameter space where the ages deviate from the established age for
Praesepe.

%   - The results
The results are shown in figure \ref{fig:cluster_results} which follows the
same layout as figure \ref{fig:sims_results}.
The top panels shows the ages inferred using an isochrone-only model, the
middle shows a gyrochrone-only model and the bottom shows an isochrone plus
gyrochrone model.
Once again, the top panel demonstrates that using an isochrone model alone
produces imprecise ages.
The middle panel shows ages recovered using only a gyrochronal model and it
reveals inaccuracies in the gyrochronology relation used here.

The bottom panel, once again, demonstrates the power of using both isochronal
and gyrochronal models together to provide a balance of precision and
accuracy.

% TEST 3: Asteroseismology
%-----------------------------------------------------------------------------
%   - The asteroseismic data
%   - The results


% \section{Discussion}
% \label{section:discussion}
% Implications
%-----------------------------------------------------------------------------
%   - Summarize the results
%   - How including the rotation period improves precision of all parameters.
%   - How will this be used/what will it be useful for?
%   - Where is it not useful?

% Future and improvements
%-----------------------------------------------------------------------------
%   - A more sophisticated gyrochronology model
%   - Mixture model, etc.

% Rolling out/future
%   - Adding new datasets/dimensions/methods.
%-----------------------------------------------------------------------------

% What if you don't have a rotation period?
% Future surveys
\section{Discussion}
\label{section:discussion}

In the previous sections we demonstrated that modeling the ages of stars using
isochrones {\it and} gyrochronology can result in more precise ages than using
isochrone fitting alone.
Isochrone fitting and gyrochronology are  complementary because gyrochronology
is more precise where isochrone fitting is less precise (on the MS) and vice
versa (at MS turn off).
% Age precision is determined by the spacing of isochrones or gyrochrones: in
% regions where iso/gyrochrones are more tightly spaced, ages will be less
% precise.
% Isochrones becone less tightly spaced (and more precise) at larger stellar
% masses and lower surface gravities.
% Gyrochrones become more tightly spaced (and less precise) at larger stellar
% masses.
%   - How will this be used/what will it be useful for?
The method we present here is available as a {\it python} package called \sd\
which allows users to infer ages from their available apparent magnitudes,
parallaxes, rotation periods and spectroscopic propertes in just a few lines
of code.
This method is applicable to an extremely large number of stars: late F, GK
and early M stars with a rotation period and broad-band photometry.
This already includes tens-of-thousands of \kepler\ and \ktwo\ stars and could
include millions more from \tess, \lsst, \wfirst, \plato, \gaia, and others in
future.
Although this method is designed for combining isochrone fitting with
gyrochronology, \sd\ can still be used without rotation periods, in
which case it will predict an isochrone-only stellar age.
% \sd\ is therefore applicable to all stars covered by the MIST isochrones:
% masses from 0.1 to 300 M$_\odot$, ages ranging from 100,000 years to longer
% than the age of the Universe, and metallicities from -4 to 0.5.
However, it is {\it optimally applicable} to stars with rotation periods,
otherwise the result will be identical to ages measured with {\tt
isochrones.py}.
%   - Where is it not useful?
However, \sd\ will often predict inaccurate ages for stars younger than around
500 million years, where stars are more likely to be rapidly rotating
outliers, and close binaries whose interactions influence their rotation
period evolution.
% These ages may still be precise even though they are inaccurate.
% Building a mixture model into \sd\ would allow these outliers to be identified
% and this is one of the main improvements to \sd\ that we plan to make in
% future.
% Since many stars with measurable rotation periods do not have precise
% spectroscopic properties, it is not always possible to tell whether a star
% falls within these permissable ranges of masses, surface gravities and rossby
% numbers.
% In addition, any given star, even if it does meet the criteria for mass,
% age, binarity, etc, may still be a rotational outlier.
Rotational outliers are often seen in clusters \citep[see \eg][]{douglas2016,
rebull2016, douglas2017, rebull2017} and many of these fall above the main
sequence, indicating that they are binaries.
When a star's age is not accurately represented by its rotation period, its
isochronal age will be in tension with its gyrochronal one, however, given the
high information content of gyrochronology, the gyrochronal age will dominate
on the MS.
% Figure \ref{fig:bimodal} shows the posterior PDF for a star with a
% misrepresentative rotation period.
% This star is rotating more rapidly than its age and mass indicate it should,
% so the gyrochronal age of this star is under-predicted.
% Situations like this are likely to arise relatively often, partly because
% rotational spin-down is not a perfect process and some unknown physical
% processes can produce outliers, and partly because misclassified giants, hot
% stars, M dwarfs or very young or very old stars will not have rotation periods
% that relate to their ages in the same way.
In addition, measured rotation periods may not always be accurate and can, in
many cases, be a harmonic of the true rotation period.
A common rotation period measurement failure mode is to measure half the true
rotation period.
The best way to prevent an erroneous or outlying rotation period from
resulting in an erroneous age measurement is to {\it allow} for outlying
rotation periods using a mixture model.
We intend to build a mixture model into \sd\ in future.
% As shown in figure \ref{fig:praesepe}, the gyrochronology model used here
% \citep{angus2015} does not provide a good fit to all available data.
% In future we intend to calibrate a new gyrochronology model that fits all
% available cluster and asteroseismic data.
% For now however, we simply warn users of these caveats and suggest that ages
% calculated using \sd\ are treated with appropriate caution.

% %   - Caveats and gotchas -- e.g. isochrones aren't 100% accurate.
% Throughout this manuscript we have referred to the `accuracy' of the
% isochronal models.
% In reality though, stellar evolution models are not 100\% accurate and
% different stellar evolution models, \eg, MIST, Dartmouth, Yonsei-Yale, etc
% will predict slightly different ages.
% The disagreement between these models varies with position on the HRD,
% but in general, ages predicted using different stellar evolution models will
% vary by around 10\%.
% We use the MIST models in our code because they cover a broader range of ages,
% masses and metallicities than the Dartmouth models.

% %   - How including the rotation period improves precision of all parameters.
% Our focus so far has been on stellar age because this is the most difficult
% stellar parameter to measure.
% However, if the age precision is improved, then the mass, \feh, distance and
% extinction precision must also be improved, since these parameters are
% strongly correlated and co-dependent in the isochronal model.
% Figure \ref{fig:mass_improvement} shows the improvement in relative precision
% of mass measurements from our simulated star sample.


% Conclusion
% \label{section:conclusion}
% Summarize. the. paper.
%-----------------------------------------------------------------------------
%   - The model
%   - The tests
%   - The results
%   - The discussion
%   - The future
\section{Conclusion}
\label{section:conclusion}

We present a statistical framework for measuring precise ages of MS stars and
subgiants by combining observables that relate, via different evolutionary
processes, to stellar age.
Specifically, we combine information used to place stars on an isochrone in an
HR diagram or CMD (\teff, \logg, observed bulk metallicity, parallax and
photometric colors) with rotation periods which are used to date stars via
their magnetic braking history (gyrochronology).
The two methods of isochrone fitting and gyrochronology are combined by taking
the product of two likelihoods: one that contains an isochronal model and the
other a gyrochronal one.
We used the MIST stellar evolution models and computed isochronal ages and
likelihoods using the {\tt isochrones.py} {\it Python} package.
The gyrochronal model is a power-law relation between rotation period, B-V
color and age, based on the functional form first introduced by
\citet{barnes2003} and later recalibrated by \citet{angus2015}.
We tested this age-dating model, called \sd, on simulated data and cluster
stars with precisely measured ages.
We found that gyrochronology predicts ages that are an order of magnitude more
precise than isochrone fitting, confirming predictions made using information
theory.
Gyrochronology and isochrone fitting are also extremely complementary:
gyrochronology supplies precise ages on the \MS\ and isochrone fitting
provides precise ages near \MS\ turn off.
\sd\ allows users to infer precise ages for MS stars and subgiants alike,
without having to first identify the age-dating method that is best for any
given star: \sd\ automatically infers the most precise possible age.
In addition, \sd\ accepts apparent magnitudes in all pass-bands covered by the
MIST isochrones which includes the Johnson-Cousins, {\it 2MASS}, \Kepler, {\it
SDSS} and \Gaia\ photometric systems.
However, we caution users that the gyrochronology model currently built into
\sd\ does not provide a good fit to all data and is not suitable for low mass
stars or those who may have ceased magnetic braking.
In the future we hope to make several improvements to the gyrochronology
relation implemented in \sd\ that will make it applicable to {\it all} MS and
subgiant stars.

The code used in this project is available as a documented {\it python}
package called \sd.
It is available for download via Github\footnote{git clone
https://github.com/RuthAngus/stardate.git} or through
PyPI\footnote{pip install stardate\_code}.
Documentation is available at https://stardate.readthedocs.io/en/latest/.
All code used to produce the figures in this paper is available at
https://github.com/RuthAngus/stardate.
\racomment{add github hash and Zenodo doi}.


% \section{Appendix}
% \label{section:appendix}
\section{Appendix}
\label{section:appendix}

\subsection*{Priors}
\label{section:priors}

We use the default priors in the {\tt isochrones.py} {\it python} package.
The prior over age is,
\begin{equation}
p(A) = \frac{\log(10) 10^{A}}{10^{10.5} - 10^8}, ~~~ 8 < A < 10.5.
\end{equation}
% where $A$, is $\log_{10}(\frac{\mathrm{Age}}{\mathrm{yrs}})$.
where $A$, is $\log_{10}(\mathrm{Age~[yrs]})$.
The prior over mass is uniform in natural-log between -20 and 20,
\begin{equation}
    p(M) = U(-20, 20)
\end{equation}
% where $M$ is $\ln(\frac{\mathrm{mass}}{M_\odot})$.
where $M$ is $\ln(\mathrm{Mass}~[M_\odot])$.
The prior over true bulk metallicity is based on the galactic metallicity
distribution, as inferred using data from the Sloan Digital Sky Survey
\racomment{citation}.
% It is based on two double-Gaussian distribution, where the halo is described as
% a broad Gaussian and the galactic disc as a narrow Gaussian.
It is the product of a Gaussian that describes the metallicity distribution
over halo stars and two Gaussians that describe the metallicity distribution
in the thin and thick disks:
\begin{eqnarray}
    p(F) =
    & H_F \frac{1}{\sqrt{2\pi\sigma_{\mathrm{halo}}^2}}
    \exp\left(-\frac{(F-\mu_{\mathrm{halo}})^2}{2\sigma_{\mathrm{halo}}}\right)
    \\ \nonumber
    & \times (1-H_F)
    \frac{1}{\xi}
    \left[\frac{0.8}{0.15}\exp\left(-\frac{(F-0.016)^2}{2\times 0.15^2}\right)
    + \frac{0.2}{0.22}\exp\left(-\frac{(F-0.15)^2}{2\times
    0.22^2}\right)\right],
\end{eqnarray}
where $H_F = 0.001$ is the halo fraction, $\mu_\mathrm{halo}$ and
$\sigma_{\mathrm{halo}}$ are the mean and standard deviation of a Gaussian
that describes a probability distribution over metallicity in the halo, and
take values -1.5 and 0.4 respectively.
% $\mu_\mathrm{disk, 1}$, $\mu_\mathrm{disk, 2}$, $\sigma_\mathrm{disk, 1}$
% and $\sigma_\mathrm{disk, 2}$ are the means and standard deviations of two
The two Gaussians inside the square brackets describe probability
distributions over metallicity in the thin and thick disks.
The values of the means and standard deviations in these Gaussians are from
\citet{casagrande2011}.
$\xi$ is the integral of everything in the square brackets from $-\infty$ to
$\infty$ and takes the value $\sim 2.507$.
% D_F = 0.8 \sigma_{\mathrm{disk, 1}} = 0.15 \mu_{\mathrm{disk, 1}} = 0.016
% \sigma_{\mathrm{disk, 2}} = 0.22 \mu_{\mathrm{disk, 2}} = 0.15
The prior over distance is,
\begin{equation}
    p(D) = \frac{3}{3000^3} D^2, ~~~ 0 < D < 3000,
\end{equation}
where D is in kiloparsecs.
Finally, the prior over extinction is uniform between zero and one,
\begin{equation}
    p(A_V) = U(0, 1).
\end{equation}


% acknowledgements
Some of the data presented in this paper were obtained from the Mikulski
Archive for Space Telescopes (MAST).
STScI is operated by the Association of Universities for Research in
Astronomy, Inc., under NASA contract NAS5-26555.
Support for MAST for non-HST data is provided by the NASA Office of Space
Science via grant NNX09AF08G and by other grants and contracts.
This paper includes data collected by the Kepler mission. Funding for the
Kepler mission is provided by the NASA Science Mission directorate.

\bibliographystyle{plainnat}
\bibliography{hz.bib}
\end{document}
