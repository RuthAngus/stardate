\section{Combining isochrone fitting with Gyrochronology: Motivation}
\label{section:motivation}

In order to demonstrate why a combination of gyrochronology and isochrone
fitting can provide more precise ages than either method used in isolation, we
calculated the information provided by each method for a range of stellar
masses, ages and evolutionary stages.

According to observations of cluster stars and the Sun, the decrease in
rotation period with time is roughly proportional to the inverse square root
of age, $\frac{dP_{\mathrm{rot}}}{dt} \propto \mathrm{Age}^{-n}$, where
n$\sim$0.5.
This corresponds to a large rate of change relative to typical rotation
period measurement uncertainties.
For example, the Sun's rotation period is currently decreasing at a rate of
around 3 days per billion years \citep[unless it has already stopped spinning
down, \eg][]{vansaders2016}, and the 1 billion year-old Sun
spun down at a rate of around 6 days per billion years.
These are relatively large changes compared with the average uncertainties on
rotation period measurements: the median rotation period uncertainty in the
\citet{mcquillan2014} catalog is around 0.1 days.
In contrast, the temperature of a K dwarf changes by about 20 K every billion
years which is small compared to typical observational uncertainties of 20-100
K.
Rotational isochrones, or `gyrochrones' provide much more {\it information}
about age than traditional isochrones.
The difference in information conveyed by rotation vs \teff\ and $L$ can be
quantified by calculating the time derivatives of a star's observables.
The rate of change of \teff\ and $L$ dictates the minimum theoretically
achievable uncertainty on an age inferred via isochrone fitting, given some
observational uncertainties.
Similarly, the rate of change of rotation period dictates the minimum
achievable uncertainty on an age inferred via gyrochronology.
In order to quantify the minimum theoretical uncertainty on ages calculated
via isochrone fitting and gyrochronology, we calculated the Fisher information
for the MIST isochrones \citep{paxton2011, paxton2013, paxton2015, dotter2016,
choi2016, paxton2018} and an empirical polynomial gyrochronology model we fit
to the Praesepe cluster.
The Fisher information quantifies the amount of information that an observable
imparts onto an unknown parameter.
In the case of isochrone fitting on a CMD using \Gaia\ data, the observables
are \Gaia\ absolute magnitude $M_G$ and color, \gcolor\ and the parameter is
age, or time, $t$.
The Fisher information is the variance of the parameter, $t$, given the
covariance of the observables and their derivatives with respect to $t$.
The inverse covariance matrix of the parameters (in this case we have just one
parameter, age or time, $t$), given the covariance matrix of the data,
${\bf y} = [M_G, G_{BP} - G_{RP}]$, is given by the following equation,
\begin{equation}
    C_{\mathrm{Age}}^{-1} = \left[\frac{d{\bf y}}{dt}\right]^T
    C_{\bf y}^{-1} \left[\frac{d{\bf y}}{dt}\right].
\end{equation}
Since we just have one parameter, $C_\mathrm{Age}^{-1}$ is a scalar, the
inverse variance of age, $\sigma_{\mathrm{Age}}^{-2}$.
In order to calculate the age uncertainty from the MIST isochrones, we
calculated numerical derivatives of $\frac{dG}{dt}$,
and $\frac{d(G_{BP} - G_{RP})}{dt}$ at every point on the MIST model grids.
We then calculated the isochronal age uncertainty, $\sigma_{\mathrm{Age}}$ at
every point on the grid.
Figure \ref{fig:iso_fisher} shows Solar-metallicity MIST isochrones, colored
by $\sigma_{\mathrm{Age}}$.

Figure \ref{fig:iso_fisher} shows the minimum theoretical absolute age
uncertainty, $\sigma_{\mathrm{Age}}$ (left panel), calculated using typical
uncertainties on \Gaia\ absolute magnitude, $M_G$, and color, \gcolor\,
represented as black errorbars in the top right corner.
The typical \Gaia\ uncertainties are $0.5$ in both $M_G$ and \gcolor.
These estimates are based on a calculation of the median uncertainty on \Gaia\
absolute G-magnitude of cool stars which is dominated by the parallax
uncertainty.
We assumed the same uncertainty on \gcolor.
The minimum uncertainty on isochronal age ranges from around 10 million years
at MS turn off (upper left yellow area) to around the age of the Universe for
K dwarfs (middle to lower-right blue area).
The minimum {\it relative} age uncertainty,
$\sigma_{\mathrm{Age}}/\mathrm{Age} \times 100$, plotted in the right-hand
panel ranges from less than 1\% for old MS turn off stars with ages
around 13 Gyr and age uncertainties less than 0.1 Gyr, up to tens-of-thousands
of percent for the youngest K and M dwarfs with unconstrained ages.

We also calculated the Fisher information for a {\it combined isochronal and
gyrochronology model.}
In this case we effectively had four observables: $M_G$ and \gcolor,
determined by the MIST isochrones; and $P_{\mathrm{rot}}$ (rotation period)
and \gcolor\ {\it again}, this time determined by the gyrochronology model.
We used a simple gyrochronology model, calibrated by fitting a fourth-order
polynomial in rotation period-\Gaia\ color space and a first order polynomical
in rotation period-age space to the 650 Myr Praesepe cluster and the Sun,
only.
This model is described in more detail later in this section.
We calculated analytic derivatives for $\frac{dP_{\mathrm{rot}}}{dt}$ and
$\left(\frac{d(G_{BP} - G_{RP})}{dt}\right)_{\mathrm{gyro}}$ and combined
these with the numerical derivatives of $\frac{dG}{dt}$ and
$\left(\frac{d(G_{BP} - G_{RP})}{dt}\right)_{\mathrm{iso}}$ in order to
calculate the total age uncertainty, $\sigma_{\mathrm{Age~(iso~\&~gyro)}}$.
Figure \ref{fig:gyro_fisher} shows Solar-metallicity MIST isochrones, colored
by $\sigma_{\mathrm{Age~(iso~\&~gyro)}}$.
The results are presented the same way as figure \ref{fig:iso_fisher} however,
figure \ref{fig:gyro_fisher} shows age uncertainties calculated using
isochrones {\it and a polynomial gyrochronology model}.
The age uncertainties were calculated using typical \Gaia\ $M_G$ and \gcolor\
uncertainties, represented as black errorbars in the top right corner, and
rotation period uncertainties of 1 day.
The minimum theoretical absolute age uncertainty inferred using gyrochronology
and isochrone fitting simultaneously, $\sigma_{\mathrm{Age~(iso~\&~gyro)}}$
(left panel of figure \ref{fig:gyro_fisher}), ranges from tens of millions of
years for stars at MS turn off to a few billion years (up to around 3 Gyr for
old G dwarfs).
The very precise ages at MS turn off are still provided by isochrone fitting
-- the incredible precision achievable with isochrone fitting at MS turn off
dominates over the precision provided by gyrochronology.
On the MS however, gyrochronology provides extremely precise ages and its
precision dominates over isochrone fitting.
The gyrochronology model used to calculate the Fisher information is not
appropriate for stars turning off the MS as it does not account for a rapid
decrease in rotation period that may be caused by the stellar radius
increasing \citep[see][]{vansaders2013}.
% However, it provides an upper limit on $\sigma_{\mathrm{Age,~gyro}}$ which is
% is, in any case, dominated by $\sigma_{\mathrm{Age,~iso}}$.
The right-hand panel of figure \ref{fig:gyro_fisher} shows the {\it relative}
age uncertainty achievable with joint isochronal and gyrochronal age
inference.
Relative age uncertainty,
$\mathrm{Age}/\sigma_{\mathrm{Age~(iso~\&~gyro)}}\times 100$ ranges from less
than 1\% at MS turn off, where isochrones provide precise ages because they
are widely spaced, to a maximum of around 30\% for young G dwarfs, where
gyrochrones are at their most tightly spaced.
The dramatic improvement in age precision seen across the MS when
gyrochronology is used provides the motivation for combining isochrone fitting
with gyrochronology.
The minimum relative age uncertainty for GKM stars on the MS is typically
around 20\% -- gyrochronology predicts precise ages for these kinds of stars.
20\% age precision for gyrochronology was also predicted in previous studies.
\citep{epstein2013}.
Gyrochronology and isochrone fitting are extremely complementary:
gyrochronology contributes most of the precision on the MS because rotation
period information dominates over color and luminosity information, however
it is not applicable to hot stars without deep convection zones
and evolved stars.
These are precisely the stars optimally fitted with isochrone models.

An important caveat of this demonstration is that this is the minimum
theoretical precision given the {\it adopted} gyrochronology model and, since
the model used for this calculation does not include intrinsic scatter (which
is particularly large for young stars), these minimum age uncertainty
calculations are over-optimistic, especially for young stars.
Similarly, our model does not account for weakened magnetic braking at old
ages \citep{vansaders2016} so is also optimistic for old dwarfs.
Still, figures \ref{fig:iso_fisher} and \ref{fig:gyro_fisher} provide an idea
of the improvement provided by gyrochronology over isochrone fitting alone.
% The Praesepe model

In order to calculate analytic derivatives of the gyrochronology model, we
fit a linear model to the Praesepe open cluster and the Sun (see figure
\ref{fig:praesepe}.
We used a three-dimensional polynomial model to predict age as a
function of \gaia\ color and rotation period for Praesepe and the Sun.
This model consists of a 4th order polynomial in logarithmic Gaia color:
\gcolor, which we write as $C_G$ for simplicity, and a 1st order polynomial (a
straight line) in logarithmic age.
For this analysis, rotation periods for Praesepe were obtained from
\citet{douglas2017} and their \gaia\ colors were obtained by crossmatching
their sky-projected positions with the \gaia\ DR2 catalog.
% We used \gcolor\ instead of (B-V) because, due to the $\sim$ billion stars
% observed by \gaia, it is now the most abundant and widely available
% photometric color.
% Our gyrochronology likelihood function is designed to compare observed
% rotation period to predicted rotation period.
% For this reason the gyrochronology model we used must predict rotation period
% as a function of age and color.
When fitting this gyrochronology model, we chose to make {\it age} the
dependent variable because the uncertainties on stellar age are much greater
than the uncertainties on rotation period.
We fit the following model to Praesepe members:
\begin{equation}
    \log_{10}(A) = a + b\log_{10}(C_G) + c\log_{10}^2(C_G) +
    d\log_{10}^3(C_G) + e\log_{10}^4(C_G) + f\log_{10}(P)
\label{eqn:gyro_age_praesepe}
\end{equation}
where $P$ is rotation period in days, $C_G$ is Gaia color, $A$ is stellar age
in years and the lower case letters are free parameters.
We adopted an age for Praesepe of 650 $\pm$ 100 million years
\citep{fossati2008, gossage2018}, a Solar age of 4.56 $\pm$ 0.01 Gyr
\citep{connelly2012}, and a Solar rotation period of 26 days \citep[][Morris
\etal, in prep]{balthasar1986, howe2000}.
The Sun's color in the Gaia color is 0.82 \citep{casagrande2018}.
We found best-fit values: $a = 7.37 \pm 0.03,~b = -1.4 \pm 0.1,~c = 5.0 \pm
0.8,~d = -34 \pm 3,~e = 66 \pm 14$, and $f = 1.49 \pm 0.02$.
This model and the data it was fit to are plotted in figure
\ref{fig:praesepe}.
To be clear, this model was only used to calculate the Fisher information and
produce figures \ref{fig:iso_fisher} and \ref{fig:gyro_fisher}, because it has
simple analytic derivatives.
Since it was only fit to a single cluster and the Sun it is not generally
applicable and was not used in any other aspects of the analysis performed in
this paper.
The gyrochronology model used throughout the rest of the analysis is described
in section \ref{section:method} and equation \ref{eqn:gyro}.
% The $f$ parameter is the inverse of the slope of the rotation period and age
% which was originally measured to be around 0.5.
% Our Praesepe and Sun-only fit results in a slightly steeper age dependence of
% around 0.67, however this value is likely be
% We inverted this relation to predict rotation period as a function of color
% and age,
% \begin{equation}
%     \log_{10}(P) = \frac{\log_{10}(A) - a - b\log_{10}(C_G) - c\log_{10}^2(C_G) -
%     d\log_{10}^3(C_G) - e\log_{10}^4(C_G)}{f}.
% \label{eqn:gyro_age_praesepe}
% \end{equation}
% Both gyrochronology models of equations \ref{eqn:gyro} and
% \ref{eqn:gyro_age_praesepe} are used to predict the ages of individual
% Praesepe stars from their rotation periods and apparent magnitudes in section
% \ref{section:results}.


\begin{figure}
  \caption{
    This figure shows Solar-metallicity MIST isochrones in \Gaia\ absolute
    G-band magnitude and Gaia $G_{BP} - G_{RP}$ color.
    In the left panel the isochrones are colored by the minimum absolute age
    uncertainty at each point on the CMD, calculated using the Fisher
    information, based on the typical uncertainties of \Gaia\ photometry
    (represented by black errorbars in the upper right).
    The Sun's position \citep{casagrande2018} is indicated with the Solar\
    symbol.
    The purple color in the top left corresponds to small age uncertainties,
    \ie\ good age precision.
    Age precision increases as stars begin to turn off the MS.
    On the MS however, particularly at low masses, the age precision is poor.
    For late K dwarfs, for example, isochrone fitting age uncertainties exceed
    the age of the Universe.
    This makes sense when you consider that typical \Gaia\ uncertaintes on $G$
    and $G_{BP} - G_{RP}$ exceed the entire width of the MS, which spans
    0.01-14 Gyrs.
    Isochrone fitting is not an appropriate age-dating method for MS stars,
    especially at low masses.
    In the right panel the isochrones are colored by the logarithmic
    {\it relative} age precision at each point in the CMD.
    Relative age uncertainties range from 100\% for the oldest MS GKM stars,
    to several thousand percent for the youngest.
    These age uncertainties were calculated using the derivatives of $G$ and
    $G_{BP} - G_{RP}$ with age, \ie\ the rate of change in a star's luminosity
    and temperature.
    For example, K dwarf temperatures increase at a rate of only around 20 K
    per billion years.
    The precision with which an age can be measured is related to the
    separation between isochrones, which indicate epochs of rapid change
    (steep gradients).\label{fig:fischer_iso}
}
  \centering
    \includegraphics[width=1\textwidth]{iso_fisher_temp.png}
\label{fig:iso_fisher}
\end{figure}

\begin{figure}
  \caption{
    As figure \ref{fig:iso_fisher}, however in this case the minimum age
    uncertainties are calculated based on isochrone fitting and gyrochronology
    {\it combined}.
    The isochrones are colored by the minimum relative age uncertainty at each
    point on the CMD, based on the typical uncertainties of Gaia photometry
    and Kepler rotation period uncertainties (assumed to be around 1 day).
    In contrast to figure \ref{fig:iso_fisher}, here the left panel shows the
    {\it logarithmic} absolute age and the right panel shows the linear
    relative age.
    The isochrones still supply precise ages at the MS turn off (purple, upper
    left) however, gyrochronology supplies precise ages on the MS (15 - 25\%
    relative precision).
    Gyrochronology and isochrone fitting complement each other and when used
    together, all subgiants and MS stars can have ages more precise than 30\%.
    \label{fig:fischer_gyro}
}
  \centering
    \includegraphics[width=1\textwidth]{gyro_fisher_temp.png}
\label{fig:gyro_fisher}
\end{figure}
