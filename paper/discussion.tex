\section{Discussion}
\label{section:discussion}

% Implications
%-----------------------------------------------------------------------------
%   - Summarize the results
%   - Precision vs accuracy.
%   - How will this be used/what will it be useful for?
%   - Where is it not useful?
%   - Caveats and gotchas -- isochrones aren't 100% accurate.
%   - How including the rotation period improves precision of all parameters.

% Future and improvements
%-----------------------------------------------------------------------------
%   - A more sophisticated gyrochronology model
%   - Mixture model, etc.

% Rolling out/future
%   - Adding new datasets/dimensions/methods.
%-----------------------------------------------------------------------------

% What if you don't have a rotation period?
% Future surveys
%=============================================================================


% Implications
%-----------------------------------------------------------------------------
%   - Summarize the results
In the previous section we demonstrated that modeling the ages of stars using
isochrones {\it and} gyrochronology can result in more precise and accurate
ages than using either isochrone fitting or gyrochronology alone.

%   - Precision vs accuracy.
Isochrone fitting and gyrochronology are extremely complementary because
gyrochronology is more precise where isochrone fitting is less precise and
vice versa.
Age precision is determined by the spacing of isochrones or gyrochrones: in
regions where iso/gyrochrones are more tightly spaced, ages will be less
precise.
Isochrones get less tightly spaced (and more precise) at larger stellar masses
and gyrochrones get more tightly spaced (and less precise) at larger stellar
masses.
Figure \ref{fig:precision} shows our simulated star sample on an H-R diagram.
Points are positioned by their true properties, not their inferred ones, and
colored by their age {\it precision} as calculated using isochrones and
gyrochronology.
Paler stars have less precise inferred ages.
% More words.
% Figure \ref{fig:accuracy} shows the simulated star sample, now colored by
% the {\it accuracy} of their inferred ages.

%   - How will this be used/what will it be useful for?
The method we present here will be useful for a large number of stars:
tens-of-thousands \kepler\ stars already and many more from
\tess, \lsst, \wfirst, \plato, \gaia, \panstarrs\ and more.
%   - Where is it not useful?
However, there are also several types of star for which gyrochronology is, in
general, {\it not} useful.
This list includes: stars with thin convective envelopes, more massive than
around 1 $M_\odot$; fully convectives stars, less massive than around 0.3
$M_\odot$; evolved stars with \logg\ less than around 4.5; stars that haven't
converged onto the rotational main sequence, \ie\ those younger than around
500 Myrs; stars who have ceased magnetic braking, \ie\ those with $Ro$ greater
than around 2.1; synchronised binaries who's rotation periods are locked to
their orbital periods; and other classes of gyrochronal outliers.
Since many stars with measurable rotation periods do not have precise
spectroscopic properties, it is not always possible to tell whether a star
falls within these permissable ranges of masses, surface gravities and rossby
numbers.
In addition, any given star, even if it does meet the criteria for mass,
evolutionary stage, age, binarity, etc, may still be a rotational outlier.
Rotational outliers are often seen in clusters \citep[see \eg][]{douglas2016,
rebull2016, douglas2017, rebull2017}.
In any case where a star's age is not truly represented by its rotation
period, its isochronal age will be in tension with its gyrochronal one.
However, given the precision of the gyrochronal technique, the gyrochronal
age may dominate over the isochronal one.
Figure \ref{fig:bimodal} shows the posterior PDF for a star with a
misrepresentative rotation period.
This star is rotating more rapidly than its age and mass indicate it should,
so the gyrochronal age of this star is under-predicted.
Situations like this are likely to arise relatively often, partly because
rotational spin-down is not a perfect process and some unknown physical
processes can produce outliers, and partly because misclassified giants, hot
stars, M dwarfs or very young or very old stars will not have rotation periods
that relate to their ages in the same way.
In addition, measured rotation periods may not always be accurate and in many
cases, due to aliasing, can be a harmonic of the true rotation period.
One of the more common rotation period measurement failure modes is to measure
half the true rotation period.
The best way to prevent an erroneous or outlying rotation period from
resulting in an erroneous age measurement is to {\it allow} for outlying
rotation periods using a mixture model.

%   - Caveats and gotchas -- e.g. isochrones aren't 100% accurate.
Throughout this manuscript we have referred to the `accuracy' of the
isochronal models.
In reality though, stellar evolution models are not 100\% accurate and
different stellar evolution models, \eg, MIST, Dartmouth, Yonsei-Yale, etc
will predict slightly different ages.
The disagreement between these models varies with position on the HR diagram,
but in general, ages predicted using different stellar evolution models will
vary by around 10\%.
We use the MIST models in our code because they cover a broader range of ages,
masses and metallicities than the Dartmouth models.

%   - How including the rotation period improves precision of all parameters.
Our focus so far has been on stellar age because this is the most difficult
stellar parameter to measure.
However, if the age precision is improved, then the mass, \feh, distance and
extinction precision must also be improved, since these parameters are
strongly correlated and co-dependent in the isochronal model.
Figure \ref{fig:mass_improvement} shows the improvement in relative precision
of mass measurements from our simulated star sample.

