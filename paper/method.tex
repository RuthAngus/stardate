\section{Method}
\label{section:method}

%   - Introduction
In this section we describe our combined isochrone fitting and gyrochronology
model.
A common approach to stellar age-dating is to make separate age predictions
using separate sets of observables.
For example, if a star's rotation period, parallax, and apparent magnitudes in
a range of bandpasses are available, it is possible to predict its age from
both gyrochronology and isochrone fitting separately.
How these two age predictions are later combined is then a difficult choice.
Is it best to average these predictions, to use the more precise of the two,
or the one believed to be more accurate?
The methodology described here provides an objective method for combining age
estimates.
There is, after all, only one age for each star.
Combining information from different models can be relatively simple, as long
as the processes being modeled; those that generated the data, are
independent.
In this case, we are combining information that relates to the burning of
hydrogen in the core (this is the process that drives the slow increase in
\teff\ and luminosity over time) with information about the magnetic braking
history of a star (the current rotation period).
We can assume that, to first order, these two processes are independent: the
hydrogen fraction in the core does not affect a star's rotation period and
vice versa.
In practise,  we can never be entirely sure that two such processes are
independent but, at least within the uncertainties, any dependence here is
unlikely to affect our results.
If this assumption is valid, the likelihoods calculated using each model can
be multiplied together.

The desired end product of this method is an estimate of the non-normalized
posterior probability density function (PDF) over the age of a star,
\begin{equation} \label{eqn:eqn1}
    p(A|{\bf m_x}, T_{\mathrm{eff}}, \log(g), \hat{F},
    P_{\mathrm{rot}}, \bar{\omega}),
\end{equation}
where $A$ is age, ${\bf m_x}$ is a vector of
apparent magnitudes in various bandpasses, \fhat\ is the {\it observed} bulk
metallicity, \prot\ is the rotation period and \pmega\ is parallax.
In order to calculate a posterior PDF over age, we must marginalize over
parameters that relate to age, but are not of interest in this study: mass
($M$), distance ($D$), V-band extinction ($A_V$) and the {\it inferred} bulk
metallicity, $F$.
The marginalization involves integrating over these extra parameters,
\begin{eqnarray} \label{eqn:bayes}
    & p(A|{\bf m_x}, T_{\mathrm{eff}}, \log(g), \hat{F},
    P_{\mathrm{rot}}, \bar{\omega})
\\ \nonumber
    & \propto \int p({\bf m_x}, T_{\mathrm{eff}}, \log(g), \hat{F},
    P_\mathrm{rot}, \bar{\omega}|
    A, M, D, A_V, F)~p(A)p(M)p(D)p(A_V)p(F)dMdDdA_VdF.
\end{eqnarray}
This equation is a form of Bayes' rule,
\begin{equation} \label{eqn:eqn2}
\mathrm{Posterior} \propto \mathrm{Likelihood} \times \mathrm{Prior},
\end{equation}
where the likelihood of the data given the model is,
\begin{equation} \label{eqn:full_likelihood}
    p({\bf m_x}, T_{\mathrm{eff}}, \log(g), \hat{F}, \bar{\omega},
    P_{\mathrm{rot}}|A, M, D, A_V, F),
\end{equation}
and the prior PDF over parameters is,
\begin{equation} \label{eqn:prior}
    p(A)p(M)p(D)p(A_V)p(F).
\end{equation}

%   - Why iso and gyro are independent.
Not all of the observables on the left of the `$|$' in the likelihood depend
on all of the parameters to the right of it.
For example, rotation period, \prot\ does not depend on V-band extinction,
$A_V$.
In our model, we make use of conditional independencies like this and use them
to factorize the likelihood.
Instead of the likelihood of equation \ref{eqn:full_likelihood},
where every observable depends on every parameter, our model can be factorized
as,
\begin{equation} \label{eqn:factorized}
    p({\bf m_x}, T_{\mathrm{eff}}, \log(g), \hat{F}, \bar{\omega},
    C_{B-V}|A, M, D, A_V, F) ~p(P_\mathrm{rot}|A, C_{B-V}),
\end{equation}
where we have introduced a new parameter, $C_{B-V}$, which is the $B-V$ color
that is often used as a mass proxy in the literature.
In our model $C_{B-V}$ is not measured but {\it inferred}: it is a latent
parameter.
We infer $C_{B-V}$ because many stars do not have a directly measured $B-V$
color.
For example, most \kepler\ stars have {\it 2MASS} photometry in J, H and K
bands and \Gaia\ photometry in $G$, $G_{BP}$ and $G_{RP}$, but do not all have
B and V band colors.
However, the gyrochronology model we use is calibrated to B-V color, not J-K
or otherwise \citep{barnes2007, mamajek2008, angus2015}.
A probabilistic graphical model (PGM) depicting the joint probability over
parameters and observables is shown in figure \ref{fig:PGM}.
It describes the conditional dependencies between parameters (in white
circles) and observables (in grey circles) with arrows leading from the causal
processes to the dependent processes.
For example, it is the mass, age, metallicity, extinction and distance that
determines the observed spectroscopic properties (\teff, \logg, \feh)
and apparant magnitudes, ${\bf m_x}$).
These parameters also determine the \cbv\ color of a star.
In turn, it is a star's age and \cbv\ color that determine its rotation
period.
Note that, written this way, stellar rotation periods do not directly depend
on stellar mass.
Mass, age and metallicity determine $C_{B-V}$, and $C_{B-V}$ along with age
determines rotation period.
The purpose of this PGM is not to depict the physical realities of stellar
evolution, it is only a visual description of the structure of the model we
use here.
Breaking up the problem this way allows us to efficiently join isochronology
and gyrochronology and infer the joint age of a star from all its observables.
It may well be that rotation period depends directly on mass and metallicity
in reality, but it is more practical for us to assume that these dependencies
are weak enough not to significantly affect the ages that we ultimately infer.

%   - The formulas
The factorization of the likelihood described in equation \ref{eqn:factorized}
and depicted in figure \ref{fig:PGM} allows us to multiply two separate
likelihood functions together: one computed using an isochronal model and one
computed using a gyrochronal model.
We assume that the probability of observing the measured observables, given
the model parameters is a Gaussian and that the observables are identically
and independently distributed.
These assumptions allow us to use Gaussian likelihood functions.
The isochronal likelihood function is,
\begin{eqnarray} \label{eqn:isochrones_only_likelihood}
    & \mathcal{L_{\mathrm{iso}}} = p({\bf m_x}, T_{\mathrm{eff}}, \log(g),
    \hat{F},
    \bar{\omega}, C_{B-V}|A, M, D,
    A_V, F) \\ \nonumber
    & = \frac{1}{\sqrt{(2\pi)^n \det(\Sigma)}}
    \exp\left( -\frac{1}{2} ({\bf O_I} - {\bf I})^T \Sigma ^{-1}
    ({\bf O_I} - {\bf I})\right),
\end{eqnarray}
where ${\bf O_I}$ is the n-dimensional vector of $n$ observables: \teff,
\logg, \fhat, \pmega, ${\bf m_x}$ (where $n$ is $4 + $ the number of
apparant magnitudes in different pass-bands that are available) and $\Sigma$
is the covariance matrix of that set of observables.
${\bf I}$ is the vector of {\it model} observables that correspond to a set of
parameters: $A$, $M$, $F$, $D$ and $A_V$, calculated using an isochrone model.
We assume there is no covariance between these observables and so this
covariance matrix consists of individual parameter variances along the
diagonal with zeros everywhere else.
The gyrochronal likelihood function is,
\begin{eqnarray} \label{eqn:gyro_likelihood}
    & \mathcal{L_{\mathrm{gyro}}} = p(P_\mathrm{rot} |A, C_{B-V}) \\ \nonumber
    & = \frac{1}{\sqrt{(2\pi) \det(\Sigma_P)}}
    \exp\left( -\frac{1}{2} ({\bf P_O} - {\bf P_P})^T \Sigma ^{-1}
    ({\bf P_O} - {\bf P_P})\right),
    % = \prod_i \frac{1}{\sqrt{2\pi}\sigma_i} \exp
    % \left(-\frac{(P_{\mathrm{obs}, i} - P_{\mathrm{pred},
    % i})^2}{2\sigma_i^2}\right),
\end{eqnarray}
% where $P_{\mathrm{obs}, i}$ is the $i$th observed rotation period,
% $P_{\mathrm{pred}, i}$ is the corresponding predicted rotation period,
% calculated from the $i$th age and $C_{B-V}$ values predicted by the isochronal
% model.
where ${\bf P_O}$ is a 1-D vector of observed rotation periods, ${\bf P_P}$ is
the vector of corresponding predicted rotation periods, calculated using the
vector of ages and $C_{B-V}$ values that correspond to the input parameters
as predicted by the isochronal model.
The full likelihood function used in our model is the product of these two
likelihood functions,
\begin{eqnarray} \label{eqn:full_likelihood}
    & \mathcal{L_{\mathrm{full}}} = \frac{1}{\sqrt{(2\pi)^n \det(\Sigma)}}
    \exp\left( -\frac{1}{2} [{\bf O_I} - {\bf I}]^T \Sigma ^{-1}
    [{\bf O_I} - {\bf I}]\right) \\ \nonumber
    & \times
    \frac{1}{\sqrt{(2\pi) \det(\Sigma_P)}}
    \exp\left( -\frac{1}{2} [{\bf P_O} - {\bf P_P}]^T \Sigma ^{-1}
    [{\bf P_O} - {\bf P_P}]\right).
\end{eqnarray}

%   - Priors
We place priors over the model parameters $A$, $M$, $F$, $D$ and $A_V$.
These priors represent our `prior beliefs' about the values these parameters
will take, before we use the data to update those beliefs via a likelihood and
produce a `posterior' belief about their values.
These priors are described in the appendix.

% Practicalities: sampling, etc.
%-----------------------------------------------------------------------------
%- isochrones.py
To calculate ${\bf I}$, the vector of predicted isochronal observables, we use
the {\tt isochrones.py} {\it python} package which has a range of
functionalities relating to isochrone fitting.
The first of the {\tt isochrones.py} functions we use is the likelihood
function of equation \ref{eqn:isochrones_only_likelihood}.
The {\tt isochrones.py} likelihood function accepts a dictionary of
observables which can, but does not {\it have} to include, all of the
following: \teff, \logg, $F$, parallax and apparent magnitudes in a range of
colors, as well as the uncertainties on all these observables.
It then calculates the residual vector $({\bf O_I} - {\bf I})$ where ${\bf
O_I}$ is the vector of observables and ${\bf I}$ is a vector of corresponding
predicted observables.
The prediction is calculated using a set of isochrones \citep[we use the MIST
models,][]{paxton2011, paxton2013, paxton2015, dotter2016, choi2016, paxton2018},
where the set of {\it model} observables that correspond
to a set of physical parameters is returned.
This requires interpolation over the model grids since, especially at high
dimensions, it is unlikely that any set of physical parameters will exactly
match a precomputed set of isochrones.
The observables that correspond to a set of physical parameters go into ${\bf
I}$ and the {\tt isochrones.py} likelihood function returns the result of
equation \ref{eqn:isochrones_only_likelihood}.
The second {\tt isochrones.py} function we use is one that predicts \cbv\ for
a given set of stellar parameters.
This color is then used to calculate the gyrochronal likelihood function of
equation \ref{eqn:gyro_likelihood}.

%   - Step-by-step description
The inference processes procedes as follows (as a reminder, we use {\it
observables} to refer to the data: \teff, \logg, etc and {\it parameters} to
refer to the model parameters: age, mass, distance, etc).
First, a set of parameters: age, mass, true bulk metallicity, distance and
extinction, as well as observables \teff, \logg, bulk metallicity, apparent
magnitudes and parallax (${\bf O_I}$) for a single star are passed to the
isochronal likelihood function, equation
\eqref{eqn:isochrones_only_likelihood}.
Then, a set of {\it model} values of \teff, \logg, bulk metallicity, apparent
magnitudes and parallax (${\bf I}$) that correspond to that set of parameters
are calculated by {\tt isochrones.py}.
The isochronal log-likelihood, $\ln(\mathcal{L}_{\mathrm{iso}})$, is then
computed for these parameter values.
The same age that was passed to the likelihood function, and the $C_{B-V}$
corresponding to it, along with the observed rotation period, are then passed
to the gyrochronal likelhood function (equation \ref{eqn:gyro_likelihood}).
The gyrochronal log-likelihood, $\ln(\mathcal{L}_{\mathrm{gyro}})$, is
computed.
The full log-likelihood is then calculated,
\begin{equation} \label{eqn:both_likelihood}
\ln(\mathcal{L}_{\mathrm{full}})
= \ln(\mathcal{L}_{\mathrm{iso}}) + \ln(\mathcal{L}_{\mathrm{gyro}}),
\end{equation}
and added to the log-prior to produce a single sample from the posterior PDF.

%   - Emcee, including assessing convergence.
When applying our model to infer the age of a star, we sampled the joint
posterior PDF over age, mass, metallicity, distance and extinction using the
affine invariant ensemble sampler, {\tt emcee} \citep{foreman-mackey2013} with
24 walkers.
Samples were drawn from the posterior PDF until 100 {\it independent} samples
are obtained.
We actively estimated the autocorrelation length, which indicates how many
steps are taken per independent sample, after every 100 steps using the
autocorrelation tool built into {\tt emcee}.
The MCMC concluded when {\it either} 100 times the autocorrelation length was
reached and the change in autocorrelation length over 100 samples was less
than 0.01, {\it or} the maximum of 100,000 samples was obtained.
This method is trivially parallelizable, since the inference process for each
star can be performed on a separate core.
The age of a single star can be inferred in around one hour on a laptop
computer.

%   - The gyrochronology models
The gyrochronology model we used to predict $P_P$ is, % $P_{\mathrm{pred}, i}$ is,
\begin{equation}
    P_\mathrm{rot} = A^\eta \alpha (C_{B-V} - \delta)^\beta,
\label{eqn:gyro}
\end{equation}
where \prot\ is rotation period in days, \cbv\ is a star's $B-V$ color, $A$ is
stellar age in Myrs and $\eta$, $\alpha$, $\beta$ and $\delta$ take values
0.55, 0.4, 0.31 and 0.45 respectively \citep{angus2015}.
This functional form was introduced by \citep{barnes2007} and the parameter
values are adopted from the recalibration performed in \citet{angus2015},
which is based on young cluster stars and old asteroseismic stars.
Because this gyochronology model is not calibrated to include stars turning
off the main sequence whose rotation slows rapidly due to their increasing
radius, we do not apply gyrochronology to stars with a MIST model \eep\
greater than 425\footnote{This number was determined by investigating the
correspondance between \eep\ and position on the HR diagram.
See the {\it Jupyter} notebook at
\url{https://github.com/RuthAngus/stardate/blob/master/paper/code/EEP_cutoff.ipynb}.
In future, \eep\ could be used as a parameter in our gyrochronology model.
}.
In addition, stars with \cbv $>$ 0.45, corresponding to a temperature around
6250 K are only modeled using isochrones, not gyrochronology.
These stars have thin convective envelopes and do not spin down substantially
over their \ms\ lifetimes so their rotation periods do not strongly predict
their ages.
However, isochrone fitting can provide relatively precise ages for these hot
stars, as well as evolved stars with large \eep s.

It was recently shown that a simple power law in age does not provide a good
fit to old asteroseismic stars \citep{angus2015, vansaders2016}.
It is hypothesized that the magnetic braking of these old stars has ceased and
cannot be modeled with a Skumanich-like spin-down law \citep{vansaders2016}.
In future, the above model could and should be updated to include a more
flexible treatment of rotation period as a function of age in order to account
for the change of slope in the relation.
Until then, this method should only be used for stars with Rossby number below
2.1 \citep{vansaders2016}, \ie\ their ratio of rotation period to convection
overturn time ($P/\tau = Ro$) does not exceed 2.1.
In this work we are chiefly concerned with introducing a new framework where
rotation periods are modeled {\it simultaneously} with isochronal features.
Although the gyrochronology models used here do not provide a good fit to all
the available data, we reiterate that no single model {\it is} able to
reproduce all the data, and that there is utility in using such a simple,
linear, empirical model like this.
Again, we are not attempting to improve gyrochronology models in this work: in
this paper we are more concerned with introducing a new approach to modeling
stellar ages, however, our method is highly flexible and modular and an
improved gyrochronology model could easily be swapped in for this one in
future.
Our model would allow a linear combination of other, {\it physical} parameters
to be used to predict age from rotation period, like $\log g$, metallicity and
mass.
In future, it may be better to model stars in physical rather than observable
parameter space.

% Although the gyrochronology model described above \citep[equation
% \ref{eqn:gyro},][]{angus2015} has been calibrated using a number of cluster
% stars, it does not provide a good fit to any individual cluster.
% No current gyrochronology model is able to capture the behavior of rotation as
% a function of color and age for individual benchmark clusters: the shape of
% this relation is different in each and current models are not flexible enough
% to capture inter-cluster differences in rotational evolution.
% For this reason, we also explored the rotational evolution of a single
% cluster, in order to produce a best-case model and demonstrate the potential
% of rotation-dating in a case where the model is perfectly accurate.
% We chose Praesepe as it is a relatively old open cluster \citep[$\sim$ 600
% Myrs][]{gossage2018}, meaning its Solar-type members have converged onto the
% rotational main sequence, and it is relatively compact on the sky so many of
% its members were observed during a single \ktwo\ campaign.
% In fact, Praesepe was repeatedly observed by \ktwo, in Campaigns 5, 16 and 18,
% however we only use rotation periods published from the analysis of Campaign 5
% in this work \citep{rebull2016}.

% % The Praesepe model
% We used a three-dimensional polynomial model to predict rotation period as a
% function of \gaia\ color and age for Praesepe and the Sun.
% This model consists of a 4th order polynomial in logarithmic Gaia color:
% $G_{Bp} - G_{Rp}$, which we write as $C_G$ for simplicity, and a 1st order
% polynomial (a straight line) in logarithmic age.
% We used \gcolor\ instead of (B-V) because, due to the $\sim$ billion stars
% observed by \gaia, it is now the most abundant and widely available
% photometric color.
% Our gyrochronology likelihood function is designed to compare observed
% rotation period to predicted rotation period.
% For this reason the gyrochronology model we used must predict rotation period
% as a function of age and color.
% However, when {\it calibrating} the gyrochronology model, we chose to make
% {\it age} the dependent variable because the uncertainties on age are much
% greater than the uncertainties on rotation period.
% Since we are using a linear model, the relation is easily invertable.
% We fitted the following model to Praesepe members:
% \begin{equation}
%     \log_{10}(A) = a + b\log_{10}(C_G) + c\log_{10}^2(C_G) +
%     d\log_{10}^3(C_G) + e\log_{10}^4(C_G) + f\log_{10}(P)
% \label{eqn:gyro_age_praesepe}
% \end{equation}
% where $P$ is rotation period in days, $C_G$ is Gaia color, $A$ is stellar age
% in years and the lower case letters are free parameters which we fitted to the
% data using linear least squares.
% We adopted an age for Praesepe of 600 million years \citep{gossage2018}, a
% Solar age of 4.56 Gyr \citep{connelly2012}, and a Solar rotation period of 26
% days \citep[][Morris \etal, in prep]{balthasar1986, howe2000}.
% The Sun's color in the Gaia color bandpasses, $G_{Bp} - G_{Rp}$, is 0.82
% \citep{casagrande2018}.
% We found best-fit values: $a = 7.37 \pm 0.03, b = -1.4 \pm 0.1, c = 5.0 \pm
% 0.8, d = -34 \pm 3, e = 66 \pm 14$, and $f = 1.49 \pm 0.02$.
% Rotation periods for Prasepe were obtained from \citet{rebull2017} and their
% \gaia\ colors were obtained by crossmatching their sky-projected positions
% with the \gaia\ DR2 catalog.
% % The $f$ parameter is the inverse of the slope of the rotation period and age
% % which was originally measured to be around 0.5.
% % Our Praesepe and Sun-only fit results in a slightly steeper age dependence of
% % around 0.67, however this value is likely be
% We inverted this relation to predict rotation period as a function of color
% and age,
% \begin{equation}
%     \log_{10}(P) = \frac{\log_{10}(A) - a - b\log_{10}(C_G) - c\log_{10}^2(C_G) -
%     d\log_{10}^3(C_G) - e\log_{10}^4(C_G)}{f}.
% \label{eqn:gyro_age_praesepe}
% \end{equation}
% Both gyrochronology models of equations \ref{eqn:gyro} and
% \ref{eqn:gyro_age_praesepe} are used to predict the ages of individual
% Praesepe stars from their rotation periods and apparent magnitudes in section
% \ref{section:results}.

% The PGM
\begin{figure}
  \caption{
A probabilistic graphical model (PGM) showing the conditional
dependencies between the parameters (white nodes) and
observables (gray nodes) in our model.
% ${\bf \theta}$ is a vector of {\it parameters}: mass, observed bulk
%     metallicity, distance and V-band extinction; and ${\bf O}$ is a vector of
%     {\it observables}: apparent magnitudes, $m_x$, effective temperature,
%     \teff, surface gravity, \logg, observed bulk metallicity, $\hat{F}$, and
%     parallax, $\bar{\omega}$.
% determined by the mass, $M$, age, $A$, distance, $D$, extinction, $A_V$
% and bulk metallicity, $F$, of a star.
${\bf \theta}$ is a vector of {\it parameters}: mass, observed bulk
    metallicity, distance and V-band extinction; and ${\bf O}$ is a vector of
    {\it observables}: apparent magnitudes, effective temperature, surface
    gravity, observed bulk metallicity, and parallax.
    The observables, ${\bf O}$, are determined by the parameters, $A$ and
    ${\bf \theta}$.
    $C_{B-V}$ is a latent parameter that is also determined by the parameters
    $A$ and ${\bf \theta}$.
    In our model, the rotation period observable, $P$, is determined {\it
    only} by the age, $A$, and color ${C_{B-V}}$ parameters.
The dependencies of observables on parameters and parameters on parameters are
    indicated by arrows that start at a `parent' node and end at the dependent
    observable, or `child' node.
In our model, rotation period does not directly depend on distance,
extinction, metallicity or mass, only age and B-V color.
This PGM is a representation of the factorized joint PDF over parameters and
observables of equation \ref{eqn:factorized}.
}
  \centering
    \includegraphics[width=.7\textwidth]{PGM}
\label{fig:PGM}
\end{figure}
