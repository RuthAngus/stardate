% \newcommand{\Teff}{$T_{\mathrm{eff}}$}
% \newcommand{\teff}{$T_{\mathrm{eff}}$}
% \newcommand{\Kepler}{{\it Kepler}}
% \newcommand{\kepler}{\Kepler}

% Motivation
% The ages of main sequence stars are difficult to infer because their outward
% appearances change subtly and slowly during their hydrogen burning lifetimes.
% % Goal
% In the era of \gaia, where precise parallaxes are available for millions of
% stars, isochrone fitting can be used to provide a constraint on stellar ages.
% In addition, for those stars with observed rotation periods, gyrochronal ages
% may also be available.

% By combining two different stellar age-dating methods and their corresponding
% observable properties that are sensitive to two different evolving processes
% in stars: core hydrogen burning and magnetic braking; it is possible to infer
% more precise and accurate ages than using either method in isolation.
By combining two different sets of observable stellar properties and dating
methods that are sensitive to two different evolving processes in stars: core
hydrogen burning and magnetic braking; it is possible to infer more precise
and accurate ages than using either method in isolation.
% isochrone fitting and gyrochronology; Multiple pieces of age-sensitive
% information may be combined to infer more precise and accurate ages than
% using Method & data In this investigation, the isochronal observables of
% main sequence stars
In this investigation, the observables of main sequence stars that are used to
trace core hydrogen burning and stellar evolution on the Hertzprung-Russell
diagram (\teff, \feh, \logg, parallax, apparent magnitude and photometric
colors) are combined with their \kepler\ rotation periods, in a Bayesian
framework, to infer stellar ages from {\it both} stellar evolution models and
gyrochronology.
% The aim of this approach is to improve upon age measurements made with either
% one of the two methods alone.
% and reveal weaknesses in one or both.
% Results
We show that incorporating rotation periods into stellar evolution models
significantly improves the precision of inferred ages.
However, since ages predicted with gyrochronology are, in general, much more
{\it precise} than isochronal ages but not necessarily more {\it accurate},
care must be taken to ensure either a) the gyrochronology relation being used
is {\it extremely} accurate, or b) its precision is relaxed by introducing a
mixture model or some intrinsic scatter or similar.
In this pilot study we do not aim to produce a new state-of-the-art dating
model, our goal is simply to explore the process of combining two
heterogeneous dating methods.
Combining methods like this can illuminate flaws in one or both, although
without ground truth it can be difficult to identify the cause of
inconsistencies.
Although calibration is not the main purpose of this exploratory investigation
and the parameters of our gyrochronology model are fixed, only a slight
modification to our algorithm would be required to perform a calibration.
Accompanying this publication is open source, packaged and documented code,
that calculates stellar ages from spectroscopic parameters and/or apparent
magnitudes, parallaxes and rotation periods.

% Since spectroscopic properties provide relatively weak age information, our
% model places greater weight on the rotation-ages of stars.
% This means that inaccuracies in any gyrochronology model we use will
% significantly propagate through to our final age inference.
% Given that rotation-dating is still poorly understood and rotational spin-down
% is a noisy process, we use a flexible gyrochronology model.

% Interpretation


% The ages of these stars are of interest because they are the most common
% type of star in the Milky Way and would thus reveal the evolution of the local
% galaxy, and because these stars are host to thousands of confirmed exoplanets.
% The ages of main sequence stars would reveal the evolving properties of
% planetary systems.
