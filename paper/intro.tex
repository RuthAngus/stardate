% Stalled. I really need to pin down the unique difference between this method
% and Jen van Saders model.

% Motivation and context
% Relevant literature
% Paper outline

% Motivation and context
%-----------------------
The formation and evolution of the Milky Way (MW) and the planetary systems
within it are two topics of significant interest to the astronomical community
today.
Both of these fields require precise and accurate ages of thousands of stars.
% In order to study the formation of the MW, it is necessary to infer the ages
% of its constituent parts, its stars.
Advances in galactic archaeology have recently been made using the ages of
red giant stars, some derived from asteroseismology and some from
spectroscopy, to probe the age distribution of the MW.
Red giants are highly luminous and can be observed to great distances, thus
providing age information on the scale of tens of kilo-parsecs.
Main sequence stars, although fainter are more numerous and their ages may
provide new insights into the formation and evolution of the Solar
neighborhood.
Stellar ages are also of great interest for studying the formation and
evolution of planetary systems.
Almost all exoplanets discovered to date orbit main sequence (MS) stars and it
is therefore the ages of MS stars that are needed to capture snapshots of
planet evolution.
Unfortunately, the very property that makes MS stars good hosts for habitable
planets also makes them difficult to date: they do not change substantially
over time.

Stellar ages provide the key to understanding the evolution of all
astrophysical objects.
For main sequence (MS) stars however, age is a difficult property to infer.
This is predominantly because hydrogen burning stars do not change appreciably
during their time spent on the MS: a star like the Sun will grow in
luminosity by around a factor of two before turning off the MS.
In addition, the Sun's temperature will only increase by around 100 K during
its $\sim$8 billion year MS lifetime.
Luminosity and temperature are therefore not sensitive proxies for age.
On the other hand, Sun's rotation period will vary by almost an order of
magnitude over its MS lifetime.
Stellar rotation periods are much more sensitive to age than luminosity or
temperature.
Ages inferred using isochrone fitting use the fact that stars get brighter and
hotter over time.
Incorporating rotation period measurements into isochrone fitting methods
provides additional information that allows for much more precise age
inference.
The models developed and calibrated by \citet{epstein2014, vansaders2015,
vansaders2016} are stellar evolution models which use rotation period as an
additional parameter and the methodology presented here is related to these
models but uses an empirically calibrated gyrochronology model, as opposed to
a physically derived one.

In addition to the difficulties imposed by the slow timescale for variability
within MS stars, different dating methods often produce inconsistent
predictions for the age of a star.
For example, an asteroseismic age will not necessarily agree with a isochronal
or rotational age.
This is in part because the underlying processes generating the evolution of
the observable properties are different and in part because our understanding
of the underlying physics is flawed or incomplete.
The various available dating methods can be categorised by the underlying
physical process they trace.
For example, evolutionary models track the radial extent of the
hydrogen-burning core and age-rotation relations model the evolving state of
the internal magnetic dynamo.
In addition, dating methods can be classified by their level of empiricism,
\ie\ the number of free parameters that need to tuned when fitting the models
to the data.
The physics behind the evolving luminosity and effective temperature of a star
as a result of core hydrogen burning is, for example, very well understood and
does not need calibrating; physics determine these models.
On the other hand, magnetic activity evolution is poorly understood and must
be calibrated using available data.
In table \ref{tab:dating_methods} we provide an overview of various dating
methods, the main observables associated with them, the underlying physics
driving the changing observables, the types of star the method applies to and
the empirical or physical nature of the model.

\subsection{Rotation-Dating}
\label{sec:rotation}

Main sequence (MS) stars comprise the majority of our galaxy but their ages
are notoriously difficult to measure.
Their positions on the HR diagram don't change significantly during their
    hydrogen burning lifetimes, a fact that is convenient for life on Earth
    but inconvenient for galactic archaeologists.
Now, due to the abundance of rotation periods for MS stars provided by Kepler
    and to-be provided by TESS, LSST and Wfirst, rotation-dating is the most
    readily available, precise method for inferring stellar ages.
Rotation-dating works well for young stars but a question mark still hangs
    over its accuracy for stars older than the Sun.
Recent results show that old \kepler\ asteroseismic stars rotate more rapidly
    than expected given their age \citep[\eg][]{Angus2015, Vansaders2016,
    Metcalfe2016}.
This has been attributed to an evolving magnetic dynamo: as stars reach a
    critical Rossby number (the ratio of rotation period to the convective
    overturn timescale), their magnetic field `switches off' and stars
    maintain a consistent rotation period after that time.
Whilst this physical explanation produces a model that fits the data, it
    is driven by observations, not theory, and other explanations could
    provide an answer.
The data sets typically used to test the age-rotation relations are highly
    heterogeneous and each set has its own detection and selection biases.
For example, asteroseismology favours quiet stars whereas rotation periods are
    easiest to measure for active stars.

% A history of rotation-dating.
The phenomenon of magnetic braking in MS stars was first observed almost fifty
years ago by \citet{Skumanich1972} who observed that the rotation periods of
the Sun and young cluster stars seemed to decay with the square-root of time.
Later, a mass-dependence was added to the relation between age and rotation
period --- less massive stars lose angular momentum faster than more massive
ones.
\citet{Kawaler1988} derived a formalism for this angular momentum loss and his
relation depended on the mass loss rate, the ....
More recently, \citet{Barnes2003} demonstrated that a simple relation could be
used to describe `gyrochronology', the method of rotation-dating, and further
works \citep[\eg][]{Barnes2007, Mamajek2008, Barnes2010, Meibom2011},
continue to demonstrate that the relation between rotation period and age
holds true while theorists \citep[\eg][]{Matt2012, Epstein2014} modify and
extend the efforts to produce physical models of this phenomenon.

Not only do a number of dating methods exists, several different models are
often available for the same dating method.
In the case of rotation-dating...

\subsection{Stellar Evolution models}
The ongoing physical processes in the core of a star is reflected externally
by an increase in luminosity and temperature.
As hydrogen is converted to helium via nuclear fusion in the core, the mass
fraction, etc, etc.
Leading to etc, etc.

