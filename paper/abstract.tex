Age is the most difficult fundamental stellar property to measure for cool
stars on the main sequence, however rotation periods can be used to predict
the ages of these stars precisely via gyrochronology.
Classical stellar evolution models are not able to predict precise stellar
ages on the main sequence, but they can provide precise ages at main sequence
turn off.
We present a method and {\it Python} package, called \sd, for inferring
stellar ages by combining two different dating methods: gyrochronology and
isochrone fitting.
This method provides ages with uncertainties no more than 20\% across the MS
and subgiant branch.
This combination of two independent age-dating methods can be applied to a
much broader range of stellar masses and evolutionary stages and provides ages
that are more precise and accurate than either method in isolation.
In this investigation, the observables of main sequence stars that are used to
trace core hydrogen burning and stellar evolution on the Hertzprung-Russell
diagram (\teff, \feh, \logg, parallax, apparent magnitude and photometric
colors) are combined with \kepler\ rotation periods, in a Bayesian framework,
to jointly infer stellar ages from both stellar evolution models/isochrone
placement and gyrochronology.
We show that incorporating rotation periods into stellar evolution models
significantly improves the precision of inferred ages on the main sequence.
However, since ages predicted with gyrochronology are, in general, much more
precise than isochronal ages, care must be taken to ensure the gyrochronology
relation being used is accurate.
In this study we did not aim to recalibrate or improve upon existing
gyrochronology models, our goal was to explore the process of combining two
independent dating methods.
However, only a slight modification to our algorithm would be required to
perform a calibration and, since the code is modular, an updated
gyrochronology model could easily replace the one used here in future.
This publication is accompanied by open source code for inferring stellar ages
for cool main sequence stars and subgiants from spectroscopic parameters
and/or apparent magnitudes, parallaxes and rotation periods.
