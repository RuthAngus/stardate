\section{Conclusion}
\label{section:conclusion}

We present a statistical framework for measuring precise ages of MS stars and
subgiants by combining observables that relate, via different evolutionary
processes, to stellar age.
Specifically, we combine information used to place stars on an isochrone in an
HR diagram or CMD (\teff, \logg, observed bulk metallicity, parallax and
photometric colors) with rotation periods which are used to date stars via
their magnetic braking history (gyrochronology).
The two methods of isochrone fitting and gyrochronology are combined by taking
the product of two likelihoods: one that contains an isochronal model and the
other a gyrochronal one.
We used the MIST stellar evolution models and computed isochronal ages and
likelihoods using the {\tt isochrones.py} {\it Python} package.
The gyrochronal model is a power-law relation between rotation period, B-V
color and age, based on the functional form first introduced by
\citet{barnes2003} and later recalibrated by \citet{angus2015}.
We tested this age-dating model, called \sd, on simulated data and cluster
stars with precisely measured ages.
We found that gyrochronology predicts ages that are an order of magnitude more
precise than isochrone fitting, confirming predictions made using information
theory.
Gyrochronology and isochrone fitting are also extremely complementary:
gyrochronology supplies precise ages on the \MS\ and isochrone fitting
provides precise ages near \MS\ turn off.
\sd\ allows users to infer precise ages for MS stars and subgiants alike,
without having to first identify the age-dating method that is best for any
given star: \sd\ automatically infers the most precise possible age.
In addition, \sd\ accepts apparent magnitudes in all pass-bands covered by the
MIST isochrones which includes the Johnson-Cousins, {\it 2MASS}, \Kepler, {\it
SDSS} and \Gaia\ photometric systems.
However, we caution users that the gyrochronology model currently built into
\sd\ does not provide a good fit to all data and is not suitable for low mass
stars or those who may have ceased magnetic braking.
In the future we hope to make several improvements to the gyrochronology
relation implemented in \sd\ that will make it applicable to {\it all} MS and
subgiant stars.

The code used in this project is available as a documented {\it python}
package called \sd.
It is available for download via Github\footnote{git clone
https://github.com/RuthAngus/stardate.git} or through
PyPI\footnote{pip install stardate\_code}.
Documentation is available at https://stardate.readthedocs.io/en/latest/.
All code used to produce the figures in this paper is available at
https://github.com/RuthAngus/stardate.
\racomment{add github hash and Zenodo doi}.
