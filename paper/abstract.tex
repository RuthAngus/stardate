% \newcommand{\Teff}{$T_{\mathrm{eff}}$}
% \newcommand{\teff}{$T_{\mathrm{eff}}$}
% \newcommand{\Kepler}{{\it Kepler}}
% \newcommand{\kepler}{\Kepler}

% Motivation
The ages of main sequence stars are difficult to infer because their outward
appearances change subtly and slowly during their hydrogen burning lifetimes.
% Goal
In the era of \gaia, where precise parallaxes are available for millions of
stars, isochrone fitting can be used to provide a constraint on stellar ages.
In addition, for those stars with observed rotation periods, gyrochronal ages
may also be available.
By combining two sets of observable stellar properties and dating methods that
are sensitive to different evolving processes in stars, it may be possible to
infer more precise and accurate ages than using either method in isolation.
% Multiple pieces of age-sensitive information may be combined to infer more
% precise and accurate ages than using
% Method & data
In this investigation, the spectroscopic properties of main sequence stars
(\teff, \feh\ and \logg) are combined with their \kepler\ rotation periods,
using a hierarchical Bayesian model, to infer their ages as predicted from {\it
both} stellar evolution models and gyrochronology.
% The aim of this approach is to improve upon age measurements made with either
% one of the two methods alone.
This is a pilot study, not aiming to produce a state-of-the-art dating model,
rather to explore the process of combining two heterogeneous dating methods.
% and reveal weaknesses in one or both.
% Results
Combining two heterogeneous dating methods can illuminate flaws in one or
both, although without ground truth it can be difficult to identify the cause
of inconsistencies.
Although calibration is not the main purpose of this exploratory investigation
and the parameters of our gyrochronology model are fixed, only a slight
modification to our algorithm would be required to perform a calibration.

% Since spectroscopic properties provide relatively weak age information, our
% model places greater weight on the rotation-ages of stars.
% This means that inaccuracies in any gyrochronology model we use will
% significantly propagate through to our final age inference.
% Given that rotation-dating is still poorly understood and rotational spin-down
% is a noisy process, we use a flexible gyrochronology model.

% Interpretation


% The ages of these stars are of interest because they are the most common
% type of star in the Milky Way and would thus reveal the evolution of the local
% galaxy, and because these stars are host to thousands of confirmed exoplanets.
% The ages of main sequence stars would reveal the evolving properties of
% planetary systems.
