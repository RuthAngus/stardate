% Motivation and context
%-----------------------

% The need for better stellar ages.
The formation and evolution of the Milky Way (MW) and the planetary systems
within it are currently two topics of interest in astronomy.
Both of these fields require precise and accurate ages of thousands of stars.
Recent advances in galactic archaeology have been made using the ages of red
giants, some calculated from asteroseismology and some from spectroscopy, to
explore the age distributions of stellar populations in the MW
\racomment{\citep[\eg][]{citation}}
Red giants are highly luminous and can be observed to great distances, thus
providing age information on the scale of tens of kilo-parsecs.
Main sequence (MS) stars on the other hand, although fainter, are more
numerous and so their ages may provide new insights into the formation and
evolution of the Solar neighborhood.
MS star ages are also of great interest for studying the formation and
evolution of planetary systems.
Almost all exoplanets discovered to date orbit MS stars and it is therefore
{\it MS star} ages that are needed to capture snapshots of planet evolution.
Unfortunately, the very property that makes MS stars good hosts for habitable
planets also makes them difficult to date: they do not change substantially
over time.

% Why are MS ages harder than red giant ages?
Unlike the spectra and photometric colors of red giants, MS star spectra and
colors do not contain a significant amount of age information because they do
not change rapidly with age.
This is represented in the spacing of isochrones on a Hertsprung-Russell (HR)
or color-magnitude diagram (CMD).
On the MS, isochrones are tightly spaced and, even with very precise
measurements of effective temperature and luminosity, the position of a MS
star on the HR diagram may be consistent with range of isochrones spanning
several billion years.
On the giant branch however, isochrones are spread further apart, so that
sufficiently precisely measured temperatures and luminosities may yield ages
that are precise to within 20\% or better.
% Typical age uncertainties of dwarfs.
\racomment{Look at typical age uncertainties from APOGEE.}
Asteroseismology can provide precise ages of both red giant and MS stars
but due to the greater abundance of observations suitable for {\it red
giant} asteroseismology, precise red giant asteroseismic ages once again
outnumber MS ages.
% The frequencies of acoustic pulsations reveal the densities of stars, and, as
% those densities are age-dependent, can provide precise stellar age
% measurements.
% \kepler, a high-precision space-based photometric survey produces light curves
% of stars that are sometimes precise enough to detect brightness fluctuations
% caused by these acoustic pulsations.
The typical periods of red giant acoustic pulsations are long (on the orders
of hours to weeks) and can be detected using \kepler's long cadence mode of
one observation per thirty minutes, which is \kepler's most common observing
mode.
In addition, the amplitudes of red giant pulsations are typically very large,
significantly greater than \kepler's photometric noise floor.
MS stars, on the other hand, oscillate with periods of just a few minutes and
the long cadence \kepler\ observations, taken once every half-hour (which is
far above the Nyquist limit) are not capable of resolving these pulsations.
Instead, to resolve the pulsation frequencies of MS stars they must be
observed in \kepler's short cadence mode of one observation every minute.
% Around two thousand stars were observed in short cadence mode during \kepler's
% original mission.
However, since the amplitude of pulsation scales with stellar radius, the
majority of stars with asteroseismic ages successfully measured using \kepler\
short-cadence observations (of which there are currently around 500) are
subgiants.
Only a small fraction of these 500 are truly on the MS.
\racomment{How many?} This may change soon however: \kepler's short cadence
light curves have recently been reprocessed and new, precise ages for the {\it
all} stars observed in short cadence mode (around 2000 in the original
\kepler\ mission) may be measured shortly.

% Introduce gyrochronology
A star like the Sun will increase in luminosity by only around a factor of two
before turning off the MS.
In addition, the Sun's temperature will only increase by around 100 K during
its $\sim$8 billion year MS lifetime.
Luminosity and temperature are not sensitive proxies for age and can also be
difficult to measure, with their precision highly sensitive to their distance
and the amount of extincting dust along the line of sight.
On the other hand, the Sun's rotation period will vary by almost an order of
magnitude over its MS lifetime.
Stellar rotation periods are much more sensitive to age than luminosity or
temperature and can be measured precisely with little dependence on distance
and none on extinction.
% Incorporating rotation period measurements into isochrone fitting methods
% provides additional information that allows for much more precise age
% inference.
Gyrochronology, an alternative dating method that uses stellar rotation
periods, has the potential to provide MS star ages that are precise to around
20\% \citep{epstein2014}.
Due to the abundance of rotation periods of MS stars already provided by
\kepler/\ktwo\ and the many more expected from future photometric surveys,
gyrochronology is one of the most readily available methods for inferring
precise stellar ages and, as such, has gained interest over the last few
years.
% How does it work?
Magnetic braking in MS stars was first observed by \citet{Skumanich1972} who,
using observations of young clusters and the Sun, found that the rotation
periods of Solar-type stars decay with the square-root of time.
It has since been established that the rotation period of a star depends, to
first order, only on its age and mass \citep[\eg][]{barnes2003}.
This means that by measuring a star's rotation period and a suitable mass
proxy (B-V color is commonly used), one can determine its age.
The convenient characteristic of stars that allows their ages to be inferred
from their {\it current} rotation periods and independently of their
primordial ones, comes from the steep dependence of spin-down rate on rotation
period \citep{kawaler1989}.
Observations of young clusters indicate that stellar angular momentum loss
rate is proportional to the cube of the angular velocity.
This means that a star spinning with high angular velocity will experience a
much greater angular momentum loss rate than a slowly spinning star.
For this reason, no matter the initial rotation period of a Sun-like star,
after around the age of the Hyades, ($\sim$ 600 Myr) stellar rotation periods
appear to converge onto a tight sequence.
After this time, the age of a star can be inferred, to first order, from its
mass and rotation period alone.

% Theoretical vs empirical gyro models
The relation between age, rotation period and mass has been studied in detail
\racomment{CITATIONS}, and several different models have been developed to
capture the rotation evolution of stars.
Some of these models are theoretical and model angular momentum loss as a
function of the stellar properties, plus the properties of the magnetic field
and stellar wind.
Others are empirical and capture the behavior of stars from a purely
observational standpoint, using simple functional forms that can reproduce the
data.
Both types of model; theoretical and empirical, must be calibrated using
observations.
Even the theoretical models are highly sensitive to stellar properties that
are not measurable: mass-loss rate and magnetic field geometry, for example.
However, despite significant advances in both types of model,
% theoretical models of stellar spin-down as well as new calibrations of
% empirical models,
the gyrochronology relations have not yet been finalized for two main reasons.
% Complex behavior
% Lack of good calibrators
In particular, they suffer from a lack of suitable calibration stars at old
ages and low masses.
These regions of parameter space are particularly important because some
evidence suggests that rotational evolution changes at old ages and low
masses.
For example, recent results show that old \kepler\ asteroseismic stars rotate
more rapidly than expected given their age \citep[\eg][]{Angus2015,
vansaders2016}.
These data can be reproduced with a model that relaxes magnetic breaking at a
critical Rossby number, $Ro$ (the ratio of rotation period to the convective
overturn timescale) of around the Solar value.
As stellar rotation periods lengthen and stars cross this $Ro$ threshold, they
maintain a constant rotation period after that time.
The gyrochronology model described in \citet{vansaders2016} and
\citet{vansaders2018} includes weakened magnetic braking after stars reach the
Solar $Ro$ threshold.

% Describe the project presented here.
The models developed and calibrated in \citet{epstein2014, vansaders2015,
vansaders2016, vansaders2018} are expensive to compute and, just as with most
isochrones and stellar evolution tracks, are usually pre-computed over a grid
of stellar parameters in order to perform tractable inference.
In order to infer an age from these models, one would effectively perform
isochrone fitting but in this case, rotation period would be added as an
additional parameter and the ages inferred would therefore be more precise and
accurate.
We present here a complementary method that combines isochrones with an
empirical gyrochronology model using a Bayesian framework.
The methodology is related to the family of models described above in that
both use a combination of rotation periods and other observable properties
that track stellar evolution on the HR diagram in concert.
A major difference is that the gyrochronology model used here is an entirely
empirically calibrated one, as opposed to a physically derived one.
One major advantage of using a physically motivated gyrochronology model over
an empirically calibrated one is the ability to rely on physics to interpolate
or extrapolate over parts of parameter space with sparse data coverage.
However, rotational spin-down is a complex process that is not yet fully
understood and currently no physical model can accurately reproduce all the
data available.
For this reason, even physically motivated gyrochronology models cannot always
be used to reliably extrapolate into unexplored parameter space.
Although we use a simple version of an empirical gyrochronology model in this
work, which, like the physical gyrochronology models, cannot yet reproduce all
the observed data, several simple modifications could be made to this model
that {\it would} produce significant improvements.
For example, including and allowing for outliers; stars with anomalously fast
or slow rotation periods, could be incorporated into our model.
Ultimately, the model we present here will provide a baseline against which
more physically motivated models, e.g. the \citet{vansaders2016} models, can
be compared.

% % Inconsistency and inaccuracy of models
% In addition to the difficulties imposed by the slow timescale for change
% within MS stars that results in poor age precision, different dating methods
% often produce inconsistent predictions for the age of a star as a result of
% model inaccuracies.
% For example, an asteroseismic age will not necessarily agree with a isochronal
% one and even isochronal ages derived from different stellar evolution models
% can be inconsistent.
% % physics driving models are wrong.
% % Not enough calibration data.
% This problem arises from a lack of calibrators with sufficiently precise
% stellar properties.

% Bayesian isochrone fitting.
% This section should contain more background about the poor uncertainties
% expected when a purely isochrone fitting technique is used.

% Sum up.
%-----------------------------------------------------------------------------

% Paper outline
%-----------------------------------------------------------------------------
